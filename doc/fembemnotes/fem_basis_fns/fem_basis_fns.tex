\chapter{Finite Element Basis Functions}

\section{Representing a One-Dimensional Field}

Consider the problem of finding a mathematical expression $\fnof{u}{x}$ to
represent a one-dimensional field \eg measurements of temperature $u$
against distance $x$ along a bar, as shown in \figref{fig:1Dfield}a.

\begin{figure}[htbp] \centering
  \input{figs/fem_basis_fns/1Dfield.pstex}
  \caption{(a) Temperature distribution $\fnof{u}{x}$ along a bar. 
    The points are the measured temperatures. (b) A least-squares polynomial
    fit to the data, showing the unacceptable oscillation between data points.}
  \label{fig:1Dfield}
\end{figure} 

One approach would be to use a polynomial expression $\fnof{u}{x} = a + bx +
cx^{2} + dx^{3} + \ldots$ and to estimate the values of the parameters
$a$, $b$, $c$ and $d$ from a least-squares fit to the data.  As the degree of the
polynomial is increased the data points are fitted with increasing accuracy
and polynomials provide a very convenient form of expression because they can
be differentiated and integrated readily. For low degree polynomials this is a
satisfactory approach, but if the polynomial order is increased further to
improve the accuracy of fit a problem arises: the polynomial can be made to
fit the data accurately, but it oscillates unacceptably between the data
points, as shown in \figref{fig:1Dfield}b.  

To circumvent this, while retaining the advantages of low degree polynomials,
we divide the bar into three subregions and use low order polynomials over
each subregion - called \emph{elements}. For later generality we also
introduce a parameter $s$ which is a measure of distance along the bar. $u$ is
plotted as a function of this arclength in \figref{fig:Tempmeas}a.
\figref{fig:Tempmeas}b shows three linear polynomials in $s$ fitted by
least-squares separately to the data in each element. 
%\todo{ref data fitting chapter}
%(we return to least-squares data fitting in \Chapref{cha:datafitting}).

\begin{figure}[htbp] \centering
  \input{figs/fem_basis_fns/Tempmeas.pstex}
  \caption{(a) Temperature measurements replotted against arclength 
    parameter $s$. (b) The $s$ domain is divided into three subdomains,
    \emph{elements}, and linear polynomials are independently fitted to the data
    in each subdomain.}
  \label{fig:Tempmeas}
\end{figure}

\section{Linear Basis Functions}

\index{Basis functions|(}\index{Basis functions!Lagrange!linear|(}A new
problem has now arisen in \figref{fig:Tempmeas}b: the piecewise linear
polynomials are not continuous in $u$ across the boundaries between elements.
One solution would be to constrain the parameters $a$, $b$, $c$ \etc to 
ensure continuity of $u$ across the element boundaries, but a better solution 
is to replace the parameters $a$ and $b$ in the first element with parameters
$u_{1}$ and $u_{2}$, which are the values of $u$ at the two ends of that
element. We then define a linear variation between these two values by
\begin{equation*}
  \fnof{u}{\xi} = \pbrac{1-\xi}u_{1} + \xi u_{2}
\end{equation*}
where $\xi (0 \leq \xi \leq 1)$ is a normalized measure of distance along the
curve.

We define 
\begin{align*}
    \lbfn{1}{\xi} &= 1-\xi  \\ 
    \lbfn{2}{\xi} &= \xi
\end{align*}
such that 
\begin{equation*}
  \fnof{u}{\xi} = \lbfn{1}{\xi} u_{1} + \lbfn{2}{\xi} u_{2}            
\end{equation*}
and refer to these expressions as the \emph{basis} functions associated with
the \emph{nodal} parameters $u_{1}$ and $u_{2}$. The basis functions
$\lbfn{1}{\xi}$ and $\lbfn{2}{\xi}$ are straight lines varying between $0$ and $1$
as shown in \figref{fig:Linearbf}.

\begin{figure}[htbp] \centering
  \input{figs/fem_basis_fns/Linearbf.pstex}
  \caption{Linear basis functions $\lbfn{1}{\xi}=1-\xi$ and 
    $\lbfn{2}{\xi}=\xi$.}
  \label{fig:Linearbf}
\end{figure}
   
It is convenient always to associate the nodal quantity $u_{n}$ with
\emph{element node} $n$ and to map the temperature $U_{\Delta}$ defined at
\emph{global node} $\Delta$ onto local node $n$ of element $e$ by using a
connectivity matrix $\fnof{\Delta}{n,e}$ \ie
\begin{equation*}
  u_{n} = U_{\fnof{\Delta}{n,e}}
\end{equation*}
where $\fnof{\Delta}{n,e}$ = global node number of local node $n$ of element $e$.
This has the advantage that the interpolation 
\begin{equation*}
  \fnof{u}{\xi} = \lbfn{1}{\xi} u_{1} + \lbfn{2}{\xi} u_{2}
\end{equation*}
holds for any element provided that $u_{1}$ and $u_{2}$ are correctly identified
with their global counterparts, as shown in \figref{fig:Relationship}.
\begin{figure}[htbp] \centering
  \input{figs/fem_basis_fns/Relationship.pstex}
  \caption{The relationship between global nodes and element nodes.}
  \label{fig:Relationship}
\end{figure}
Thus, in the first element
\begin{equation}
  \fnof{u}{\xi} = \lbfn{1}{\xi}u_{1} + \lbfn{2}{\xi}u_{2}
  \label{eqn:1stelement}
\end{equation}
with $u_{1}=U_{1}$ and $u_{2}=U_{2}$.

In the second element $u$ is interpolated by
\begin{equation}
  \fnof{u}{\xi}  =  \lbfn{1}{\xi}u_{1} + \lbfn{2}{\xi}u_{2}
  \label{eqn:2ndelement}
\end{equation}
with $u_{1}=U_{2}$ and $u_{2}=U_{3}$, since the parameter $U_{2}$ is shared
between the first and second elements the temperature field $u$ is implicitly
continuous. Similarly, in the third element $u$ is interpolated by
\begin{equation}
  \fnof{u}{\xi} = \lbfn{1}{\xi}u_{1} + \lbfn{2}{\xi}u_{2}
  \label{eqn:3rdelement}
\end{equation}
with $u_{1}=U_{3}$ and $u_{2}=U_{4}$, with the parameter $U_{3}$ being shared
between the second and third elements. \Figref{fig:Weighting} shows the
temperature field defined by the three interpolations
\bref{eqn:1stelement}--\bref{eqn:3rdelement}. \index{Basis functions!Lagrange!linear|)}

\begin{figure}[htbp] \centering
  \input{figs/fem_basis_fns/Tmfit.pstex}
  \caption{Temperature measurements fitted with nodal parameters and linear 
    basis functions.  The fitted temperature field is now continuous across
    element boundaries.}
  \label{fig:Tmfit}
\end{figure}

\section{Basis Functions as Weighting Functions}

It is useful to think of the basis functions as weighting functions on the
nodal parameters.  Thus, in element 1
\begin{equation*}
  \text{at } \xi=0 \qquad \fnof{u}{0} = \pbrac{1-0}u_{1} + 0u_{2} = u_{1}
\end{equation*}
which is the value of $u$ at the left hand end of the element and has no
dependence on $u_{2}$
\begin{equation*}
  \text{at } \xi=\frac14 \qquad   \fnof{u}{\frac14} = \pbrac{1-\frac14} u_{1} +
  \frac14 u_{2} = \frac34 u_{1}+\frac14 u_{2}
\end{equation*}
which depends on $u_{1}$ and $u_{2}$, but is weighted more towards $u_{1}$
than $u_{2}$
\begin{equation*}
  \text{at } \xi=\frac12 \qquad \fnof{u}{\frac12} = \pbrac{1-\frac12} u_{1}+
  \frac12 u_{2} = \frac12 u_{1}+\frac12 u_{2}
\end{equation*}
which depends equally on $u_{1}$ and $ u_{2}$
\begin{equation*}
  \text{at } \xi=\frac34   \qquad \fnof{u}{\frac34} = \pbrac{1-\frac34} u_{1}
  + \frac34 u_{2} = \frac14 u_{1}+\frac34 u_{2}
\end{equation*}
which depends on $u_{1}$ and $u_{2}$ but is weighted more towards $u_{2}$
than $u_{1}$
\begin{equation*}
    \text{at } \xi=1   \qquad \fnof{u}{1} = \pbrac{1- 1}u_{1} +{1}u_{2} = u_{2}
\end{equation*}
which is the value of $u$ at the right hand end of the region and has no
dependence on $u_{1}$.

Moreover, these weighting functions can be considered as \emph{global} 
functions, as shown in \figref{fig:Weighting}, where the weighting function
$w_{n}$ associated with global node $n$ is constructed from the basis functions 
in the elements adjacent to that node.

\begin{figure}[htbp] \centering
  \input{figs/fem_basis_fns/Weighting.pstex}
  \caption{(a) $\ldots$ (d) The weighting functions $w_{n}$ associated with 
    the global nodes $n=1\ldots4$, respectively. Notice the linear fall off 
    in the elements adjacent to a node.  Outside the immediately adjacent 
    elements, the weighting functions are defined to be zero.}
  \label{fig:Weighting}
\end{figure}

For example, $w_{2}$ weights the global parameter $U_{2}$ and the influence of
$U_{2}$ falls off linearly in the elements on either side of node 2.

We now have a continuous piecewise parametric description of the temperature
field $\fnof{u}{\xi}$ but in order to define $\fnof{u}{x}$ we need to define
the relationship between $x$ and $\xi$ for each element. A convenient way to
do this is to define $x$ as an interpolation of the nodal values of $x$.

For example, in element 1
\begin{equation}
  \fnof{x}{\xi} = \lbfn{1}{\xi}x_{1} + \lbfn{2}{\xi}x_{2}
\end{equation}
and similarly for the other two elements. The dependence of temperature on $x$,
$\fnof{u}{x}$, is therefore defined by the parametric expressions
\begin{align*}
  \fnof{u}{\xi} &= \dsuml{n}{} \lbfn{n}{\xi} u_{n} \\
  x(\xi) &= \dsuml{n}{} \lbfn{n}{\xi} x_{n}
\end{align*}
where summation is taken over all element nodes (in this case only $2$) and
the parameter $\xi$ (the ``element coordinate'') links temperature $u$ to
physical position $x$. $\fnof{x}{\xi}$ provides the mapping between the
mathematical space $0 \leq \xi \leq 1$ and the physical space $x_{1} \leq x \leq
x_{2}$, as illustrated in \figref{fig:xandurel}.

\begin{figure}[htbp] \centering
  \input{figs/fem_basis_fns/xandurel.pstex}
  \caption{Illustrating how $x$ and $u$ are related through the normalized 
    element coordinate $\xi$. The values of $\fnof{x}{\xi}$ and $\fnof{u}{\xi}$ are 
    obtained from a linear interpolation of the nodal variables and then 
    plotted as $\fnof{u}{x}$. The points at $\xi=0.2$ are emphasized.}
  \label{fig:xandurel}
\end{figure}

\section{Quadratic Basis Functions}

\index{Basis functions!Lagrange!quadratic|(}The essential property of the 
basis functions defined above is that the basis function associated with a 
particular node takes the value of $1$ when evaluated at that node and is zero at
every other node in the element (only one other in the case of linear basis 
functions). This ensures the linear independence of the basis functions. 
It is also the key to establishing the form of the basis functions for higher order
interpolation.  For example, a quadratic variation of $u$ over an element
requires three nodal parameters $u_{1}$, $u_{2}$ and $u_{3}$
\begin{equation}
  \fnof{u}{\xi} = \lbfn{1}{\xi}u_{1} + \lbfn{2}{\xi}u_{2} + \lbfn{3}{\xi}u_{3}  
  \label{eqn:Qbf}
\end{equation}
The quadratic basis functions are shown, with their mathematical expressions,
in \figref{fig:1DQbf}. Notice that since $\lbfn{1}{\xi}$ must be zero at
$\xi=0.5$ (node $2$), $\lbfn{1}{\xi}$ must have a factor $\pbrac{\xi-0.5}$
and since it is also zero at $\xi=1$ (node $3$), another factor is
$\pbrac{\xi-1}$.  Finally, since $\lbfn{1}{\xi}$ is $1$ at $\xi=0$ (node $1$)
we have $\lbfn{1}{\xi} = 2\pbrac{\xi-1}\pbrac{\xi-0.5}$.  Similarly for the
other two basis functions.

\begin{figure}[htbp] \centering
  \input{figs/fem_basis_fns/1DQbf.pstex}
  \caption{One-dimensional quadratic basis functions.}
  \label{fig:1DQbf}
\end{figure}
\index{Basis functions!Lagrange!quadratic|)}

\section{Two- and Three-Dimensional Elements}

\index{Basis functions!Lagrange!bilinear|(}Two-dimensional bilinear 
basis functions are constructed from the products of the above 
one-dimensional linear functions as follows

Let   
\begin{equation*}
  \fnof{u}{\xione,\xitwo} = \lbfn{1}{\xione,\xitwo}u_{1} +
  \lbfn{2}{\xione,\xitwo}u_{2} +
  \lbfn{3}{\xione,\xitwo}u_{3} + \lbfn{4}{\xione,\xitwo}u_{4}
\end{equation*}
where  
\begin{equation}
\begin{split}
    \lbfn{1}{\xione,\xitwo} &= \pbrac{1- \xione}\pbrac{1- \xitwo} \\
    \lbfn{2}{\xione,\xitwo} &= \xione\pbrac{1- \xitwo} \\ 
    \lbfn{3}{\xione,\xitwo} &= \pbrac{1- \xione}\xitwo \\
    \lbfn{4}{\xione,\xitwo} &= \xione\xitwo
  \end{split}
  \label{eqn:2,3DE}
\end{equation}

Note that
\lbfn{1}{\xione,\xitwo} = \lbfn{1}{\xione}\lbfn{1}{\xitwo} 
where \lbfn{1}{\xione} and \lbfn{1}{\xitwo} are the one-dimensional linear basis
functions. Similarly, \lbfn{2}{\xione,\xitwo} =
\lbfn{2}{\xione}\lbfn{1}{\xitwo} \ldots \etc

These four bilinear basis functions are illustrated in \figref{fig:2Dbbf}.
\begin{figure}[htbp] \centering
  \input{figs/fem_basis_fns/2Dbbf.pstex}
  \caption{Two-dimensional bilinear basis functions.}
  \label{fig:2Dbbf}
\end{figure}

Notice that $\lbfn{n}{\xione,\xitwo}$ is $1$ at node $n$ and zero at the
other three nodes.  This ensures that the temperature
$\fnof{u}{\xione,\xitwo}$ receives a contribution from each nodal parameter
$u_{n}$ weighted by $\lbfn{n}{\xione,\xitwo}$ and that when
$\fnof{u}{\xione,\xitwo}$ is evaluated at node $n$ it takes on the value
$u_{n}$.

As before the geometry of the element is defined in terms of the node
positions $\pbrac{x_{n},y_{n}}$, $n=1,\ldots,4$ by
\begin{align*}
  x &= \dsuml{n}{} \lbfn{n}{\xione,\xitwo} x_{n} \\ 
  y &= \dsuml{n}{} \lbfn{n}{\xione,\xitwo} y_{n}
\end{align*}
which provide the mapping between the mathematical space $\pbrac{\xione, \xitwo}$
(where $0 \leq \xione,  \xitwo \leq 1$) and the physical space $\pbrac{x,y}$.

%\begin{example}{Defining a 2D bilinear finite element mesh}
%  {Defining a 2D bilinear finite element mesh}

%  To define a 2D bilinear finite element mesh run the CMISS example number
%  $111$.  The nodes should be position as shown in \figref{fig:example111_node}.
%  After defining elements the mesh should appear like the one shown in
%  \figref{fig:example111_elem}

%  \begin{figure}[htbp] \centering
%    \input{figs/fem_basis_fns/egdiag1.pstex}
%    \caption{Node positions for example $111$.}
%    \label{fig:example111_node}
%  \end{figure}
%   \begin{figure}[hbtp] \centering
%    \input{figs/fem_basis_fns/egdiag2.pstex}
%    \caption{2D bilinear finite element mesh for example $111$.}
%    \label{fig:example111_elem}
%  \end{figure} 
%  \label{xmp:Def2D}
%\end{example}

%\begin{example}{Refining a mesh}
%  {Refining a mesh} 
%  To refine a mesh run the CMISS example $113$. After the first refine the
%  mesh should appear like the one shown in \figref{fig:example113}.  
%\begin{figure}[htbp] \centering
%    \input{figs/fem_basis_fns/egdiag3.pstex}
%    \caption{Refined mesh.}
%    \label{fig:example113}
%  \end{figure}
%  \label{xmp:Refining}
%\end{example}

\index{Basis functions!Lagrange!bilinear} 

Higher order 2D basis functions can be similarly constructed from products
of the appropriate 1D basis functions. For example, a six-noded
(see \figref{fig:6node}) quadratic-linear element (quadratic in $\xione$ and
linear in $\xitwo$) would have 
\begin{equation*} u=\dsuml{n=1}{6}
  \lbfn{n}{\xione,\xitwo}u_{n}
\end{equation*}
where
  \begin{alignat}{2}
    \lbfn{1}{\xione,\xitwo} &=
    2\pbrac{\xione-1}\pbrac{\xione-0.5}\pbrac{1-\xitwo} &\qquad
    \lbfn{2}{\xione,\xitwo} &= 4\xione\pbrac{1-\xione}\pbrac{1-\xitwo} \\
    \lbfn{3}{\xione,\xitwo} &=
    2\xione\pbrac{\xione-0.5}\pbrac{1-\xitwo} &\qquad
    \lbfn{4}{\xione,\xitwo} &=
    2\pbrac{\xione-1}\pbrac{\xione-0.5}\xitwo \\ 
    \lbfn{5}{\xione,\xitwo} &= 4\xione\pbrac{1-\xione}\xitwo &\qquad
    \lbfn{6}{\xione,\xitwo} &= 2\xione\pbrac{\xione-0.5}\xitwo
  \label{eqn:QLE}
  \end{alignat}



\begin{figure}[htbp] \centering
  \input{figs/fem_basis_fns/6node.pstex}
  \caption{A $6$-node quadratic-linear element (node numbers circled).}
  \label{fig:6node}
\end{figure} 

Three-dimensional basis functions are formed similarly, \eg a trilinear
element basis has eight nodes (see \figref{fig:8nte}) with basis functions
\begin{alignat}{2}
        \lbfn{1}{\xione,\xitwo,\xithree} &=
        \pbrac{1-\xione}\pbrac{1-\xitwo}\pbrac{1-\xithree} 
        &\qquad \lbfn{2}{\xione,\xitwo,\xithree}
        &=\xione\pbrac{1-\xitwo}\pbrac{1-\xithree} \\
        \lbfn{3}{\xione,\xitwo,\xithree} &=\pbrac{1-\xione}\xitwo\pbrac{1-\xithree} 
        &\qquad \lbfn{4}{\xione,\xitwo,\xithree} &=\xione\xitwo\pbrac{1-\xithree} \\
        \lbfn{5}{\xione,\xitwo,\xithree} &=\pbrac{1-\xione}\pbrac{1-\xitwo}\xithree 
        &\qquad \lbfn{6}{\xione,\xitwo,\xithree} &=\xione\pbrac{1-\xitwo}\xithree \\
        \lbfn{7}{\xione,\xitwo,\xithree} &=\pbrac{1-\xione}\xitwo\xithree          
        &\lbfn{8}{\xione,\xitwo,\xithree} &=\xione\xitwo\xithree
        \label{eqn:trilinear}
\end{alignat}


\begin{figure}[htbp] \centering
  \input{figs/fem_basis_fns/8nte.pstex}
  \caption{An $8$-node trilinear element.}
  \label{fig:8nte}
\end{figure}

%\begin{example}{Defining a quadratic-linear element mesh}
%  {Defining a quadratic-linear element mesh.} 
%  To define a quadratic-linear element run the cmiss example $115$.
%  \label{xmp:Defining_quadlin}
%\end{example}

%\begin{example}{Defining a 3D trilinear element mesh}
%  {Defining a 3D trilinear element mesh} 
%  To define a 3D trilinear element run CMISS example $121$.
%  \label{xmp:Def3D}
%\end{example}

\section{Higher Order Continuity}
\index{Basis functions!Hermite|(}\index{Basis functions!Hermite!cubic|(}
All the basis functions mentioned so far are \emph{Lagrange}\footnote{Joseph-Louis
Lagrange (1736-1813).} basis functions\index{Basis functions!Lagrange} and provide continuity of $u$ across element boundaries 
but not higher order continuity.  Sometimes it is desirable to use basis
functions which also preserve continuity of the derivative of $u$ with respect
to $\xi$ across element boundaries. A convenient way to achieve this is by
defining two additional nodal parameters $\pbrac{\dby{u}{\xi}}_{n}$.  The basis
functions are chosen to ensure that 
\begin{equation*}
 \evalat{\dby{u}{\xi}}{\xi=0} =\pbrac{\dby{u}{\xi}}_{1} = u'_{1} \text{ and }
 \evalat{\dby{u}{\xi}}{\xi=1} =\pbrac{\dby{u}{\xi}}_{2} = u'_{2}
\end{equation*}
and since $u_n$ is shared between adjacent elements derivative continuity is 
ensured.  Since the number of element parameters is 4 the basis functions must
be cubic in $\xi$.  To derive these
cubic \emph{Hermite}\footnote{Charles Hermite (1822-1901).} basis functions
let
\begin{align*}
  \fnof{u}{\xi} &= a + b\xi + c\xi^2 + d\xi^3 , \\
  \dby{u}{\xi} &= b + 2c\xi + 3d\xi^2 ,
\end{align*}
and impose the constraints    
\begin{alignat*}{2}
  \fnof{u}{0} &= a \quad &&= u_{1} \\ 
  \fnof{u}{1} &= a + b + c + d \quad &&= u_{2} \\
  \fnof{\dby{u}{\xi}}{0} &= b \quad &&= u_{1}^{'}\\
  \fnof{\dby{u}{\xi}}{1} &= b + 2c + 3d \quad &&= u_{2}^{'}
\end{alignat*}
These four equations in the four unknowns $a$, $b$, $c$ and $d$ are solved to
give
\begin{align*}
    a &= u_{1} \\   
    b &= u'_{1} \\
    c &= 3u_{2} - 3u_{1} - 2 u'_{1} -  u'_{2}   \\
    d &= u'_{1} + u'_{2}  + 2u_{1}  - 2u_{2}
\end{align*}
Substituting $a$, $b$, $c$ and $d$ back into the original cubic then gives 
\begin{equation*}
  \fnof{u}{\xi} =u_{1} + u'_{1}\xi + \pbrac{3u_{2} - 3u_{1} - 2u'_{1} -
    u'_{2}}\xi^2 + \pbrac{u'_{1} + u'_{2}  + 2u_{1} - 2u_{2}}\xi^3 
\end{equation*}
or, rearranging,
\begin{equation}
  \fnof{u}{\xi} =\chbfn{1}{0}{\xi}u_{1} + \chbfn{1}{1}{\xi}u'_{1} +
  \chbfn{2}{0}{\xi}u_{2} + \chbfn{2}{1}{\xi}u'_{2} 
  \label{eqn:Hoc}
\end{equation}
where the four cubic Hermite basis functions are drawn in \figref{fig:cubic}.

\begin{figure}[htbp] \centering
  \input{figs/fem_basis_fns/cubic.pstex}
  \caption{Cubic Hermite basis functions.}
  \label{fig:cubic}
\end{figure}

One further step is required to make cubic Hermite basis functions useful in
practice. The derivative $\pbrac{\dby{u}{\xi}}_{n}$ defined at node $n$
is dependent upon the element $\xi$-coordinate in the two adjacent elements.
It is much more useful to define a global node derivative
 $\pbrac{\dby{u}{s}}_{n}$ where $s$ is arclength and then use
\begin{equation}
  \pbrac{\dby{u}{\xi}}_{n}=\pbrac{\dby{u}{s}}_{\fnof{\Delta}{n,e}} \cdot
  \pbrac{\dby{s}{\xi}}_{n} 
  \label{eqn:cubichbf}
\end{equation}
where $\pbrac{\dby{s}{\xi}}_{n}$ is an element \emph{scale factor} which scales
the arclength derivative of global node $\Delta$ to the $\xi$-coordinate
derivative of element node $n$. Thus $\dby{u}{s}$ is constrained to be
continuous across element boundaries rather than $\dby{u}{\xi}$.\index{Basis
  functions!Hermite!cubic|)} \index{Basis functions!Hermite!bicubic|)}A two-
dimensional bicubic Hermite basis requires four derivatives per node
\begin{equation*}
  u, \delby{u}{\xione}, \delby{u}{\xitwo} \text{ and }
  \deltwoby{u}{\xione}{\xitwo}
\end{equation*}
The need for the second-order cross-derivative term can be explained as
follows; If $u$ is cubic in $\xione$ and cubic in $\xitwo$, then
$\delby{u}{\xione}$ is quadratic in $\xione$ and cubic in $\xitwo$ , and
$\delby{u}{\xitwo}$ is cubic in $\xione$ and quadratic in $\xitwo$ . Now
consider the side 1--3 in \figref{fig:interpol}. The cubic variation of $u$
with $\xitwo$ is specified by the four nodal parameters $u_{1}$,
$\pbrac{\delby{u}{\xitwo}}_{1}$, $u_{3}$ and $\pbrac{\delby{u}{\xitwo}}_{3}$.
But since $\delby{u}{\xione}$ (the normal derivative) is also cubic in
$\xitwo$ along that side and is entirely independent of these four parameters,
four additional parameters are required to specify this cubic. Two of these
are specified by $\pbrac{\delby{u}{\xione}}_{1}$ and
$\pbrac{\delby{u}{\xione}}_{3}$, and the remaining two by
$\pbrac{\deltwoby{u}{\xione}{\xitwo}}_{1}$ and
$\pbrac{\deltwoby{u}{\xione}{\xitwo}}_{3}$.

\begin{figure}[htbp] \centering
  \input{figs/fem_basis_fns/interpol.pstex}
  \caption{Interpolation of nodal derivative $\delby{u}{\xione}$ along side 
    1--3.}
  \label{fig:interpol}
\end{figure}
The bicubic interpolation of these nodal parameters is given by
\begin{gather}
  \begin{aligned}
    \fnof{u}{\xione,\xitwo} &= \chbfn{1}{0}{\xione}\chbfn{1}{0}{\xitwo}u_{1} +
    \chbfn{2}{0}{\xione}\chbfn{1}{0}{\xitwo}u_{2} \\ 
    &+ \chbfn{1}{0}{\xione}\chbfn{2}{0}{\xitwo}u_{3} +
    \chbfn{2}{0}{\xione}\chbfn{2}{0}{\xitwo}u_{4} \\ &+
    \chbfn{1}{1}{\xione}\chbfn{1}{0}{\xitwo}\pbrac{\delby{u}{\xione}}_{1} +
    \chbfn{2}{1}{\xione}\chbfn{1}{0}{\xitwo}\pbrac{\delby{u}{\xione}}_{2} \\ &+
    \chbfn{1}{1}{\xione}\chbfn{2}{0}{\xitwo}\pbrac{\delby{u}{\xione}}_{3} +
    \chbfn{2}{1}{\xione}\chbfn{2}{0}{\xitwo}\pbrac{\delby{u}{\xione}}_{4} \\ &+
    \chbfn{1}{0}{\xione}\chbfn{1}{1}{\xitwo}\pbrac{\delby{u}{\xitwo}}_{1} +
    \chbfn{2}{0}{\xione}\chbfn{1}{1}{\xitwo}\pbrac{\delby{u}{\xitwo}}_{2} \\ &+
    \chbfn{1}{0}{\xione}\chbfn{2}{1}{\xitwo}\pbrac{\delby{u}{\xitwo}}_{3} +
    \chbfn{2}{0}{\xione}\chbfn{2}{1}{\xitwo}\pbrac{\delby{u}{\xitwo}}_{4} \\ &+
    \chbfn{1}{1}{\xione}\chbfn{1}{1}{\xitwo}
    \pbrac{\deltwoby{u}{\xione}{\xitwo}}_{1} +
    \chbfn{2}{1}{\xione}\chbfn{1}{1}{\xitwo}
    \pbrac{\deltwoby{u}{\xione}{\xitwo}}_{2} \\ &+ 
    \chbfn{1}{1}{\xione}\chbfn{2}{1}{\xitwo}
    \pbrac{\deltwoby{u}{\xione}{\xitwo}}_{3} +
    \chbfn{2}{1}{\xione}\chbfn{2}{1}{\xitwo}
    \pbrac{\deltwoby{u}{\xione}{\xitwo}}_{4}  
  \end{aligned}
  \label{eqn:bicubic}
\end{gather}
where
\begin{gather}
  \begin{aligned}
    \chbfn{1}{0}{\xi} \quad &= \quad 1-3\xi^2+2\xi^3 \\ 
    \chbfn{1}{1}{\xi} \quad &= \quad \xi\pbrac{\xi -1}^2 \\
    \chbfn{2}{0}{\xi} \quad &= \quad \xi^2\pbrac{3-2\xi} \\
    \chbfn{2}{1}{\xi} \quad &= \quad \xi^2\pbrac{\xi -1}
  \end{aligned}
\end{gather}
are the one-dimensional cubic Hermite basis functions (see \figref{fig:cubic}).

As in the one-dimensional case above, to preserve derivative continuity in
physical x-coordinate space as well as in $\xi$-coordinate space the global
node derivatives need to be specified with respect to physical arclength.
There are now two arclengths to consider: $s_{1}$, measuring arclength along
the $\xione$-coordinate, and $s_{2}$, measuring arclength along the 
$\xitwo$-coordinate. Thus
\begin{gather}
  \begin{aligned}
    \pbrac{\delby{u}{\xione}}_{n} &=
    \pbrac{\delby{u}{s_{1}}}_{\fnof{\Delta}{n,e}} \cdot
    \pbrac{\delby{s_{1}}{\xione}}_{n} \\ \pbrac{\delby{u}{\xitwo}}_{n} &=
    \pbrac{\delby{u}{s_{2}}}_{\fnof{\Delta}{n,e}} \cdot
    \pbrac{\delby{s_{2}}{\xitwo}}_{n} \\ 
    \pbrac{\deltwoby{u}{\xione}{\xitwo}}_{n} &=
    \pbrac{\deltwoby{u}{s_{1}}{s_{2}}}_ {\fnof{\Delta}{n,e}} \cdot
    \pbrac{\dby{s_{1}}{\xione}}_{n} \cdot \pbrac{\dby{s_{2}}{\xitwo}}_{n}
  \end{aligned}
  \label{eqn:Hbf}
\end{gather}
where $\pbrac{\dby{s_{1}}{\xione}}_{n}$ and $\pbrac{\dby{s_{2}}{\xitwo}}_{n}$
are element \emph{scale factors} which scale the arclength derivatives of
global node $\Delta$ to the $\xi$-coordinate derivatives of element node $n$.

The bicubic Hermite basis is a powerful shape descriptor for curvilinear
surfaces.  \figref{fig:surface} shows a four element bicubic Hermite
surface in 3D space where each node has the following twelve parameters
\begin{equation*}
   x,\/ \delby{x}{s_{1}},\/ \delby{x}{s_{2}},\/
   \deltwoby{x}{s_{1}}{s_{2}},\/y,\/ \delby{y}{s_{1}},\/\delby{y}{s_{2}},\/
   \deltwoby{y}{s_{1}}{s_{2}},\/  z,\/
   \delby{z}{s_{1}},\/ \delby{z}{s_{2}} \text{ and } \deltwoby{z}{s_{1}}{s_{2}}
\end{equation*}

\begin{figure}[htbp] \centering
  \input{figs/fem_basis_fns/surface.pstex}
  \caption{A surface formed by four bicubic Hermite elements.}
  \label{fig:surface}
\end{figure}


%\begin{example}{Defining a 2D cubic Hermite-linear finite element mesh}
%  {Defining a 2D cubic Hermite-linear finite element mesh}  
%  To define a 2D cubic Hermite-linear finite element mesh run example 114
%  \label{sec:Def2Db}
%\end{example}

\index{Basis functions!Hermite!bicubic|)}\index{Basis functions!Hermite|)}

\section{Triangular Elements} \index{Triangular elements|(}
Triangular elements cannot use the $\xione$ and $\xitwo$ coordinates defined 
above for \emph{tensor product} elements (\ie
two- and three- dimensional elements whose basis functions are formed as the
product of one-dimensional basis functions). The natural coordinates for
triangles are based on area ratios and are called \emph{Area Coordinates}
\index{Area Coordinates}.  
Consider the ratio of the area formed from the points $2$, $3$ and 
 $\fnof{P}{x,y}$ in \figref{fig:areacoord} to the total area of the triangle
\begin{figure}[htbp] \centering
  \input{figs/fem_basis_fns/areacoord.pstex}
  \caption{Area coordinates for a triangular element.}
  \label{fig:areacoord}
\end{figure}

\begin{equation*}
  L_{1}=\dfrac{\text{Area $<P23>$}}{\text{Area $<123>$}}=\frac12
    \begin{vmatrix}
            1 & x & y\\
            1 & x_{2} & y_{2}\\
            1 & x_{3} & y_{3}
    \end{vmatrix}
  / \Delta = \pbrac{a_{1} + b_{1}x + c_{1}y}/ \pbrac{2 \Delta}
\end{equation*}
where $ \Delta = \frac12             
    \begin{vmatrix}
            1 & x_{1} & y_{1}\\
            1 & x_{2} & y_{2}\\
            1 & x_{3} & y_{3}
    \end{vmatrix}$ 
    is the area of the triangle with vertices $123$, and $a_{1} = x_{2} y_{3}
    - x_{3} y_{2}, b_{1} = y_{2} - y_{3}, c_{1} = x_{3} - x_{2}$.

  Notice that $L_{1}$ is linear in $x$ and $y$. Similarly, area coordinates for
  the other two triangles containing $P$ and two of the element vertices are
\begin{equation*}
  L_{2}=\dfrac{\text{Area $<P13>$}}{\text{Area $<123>$}}=\frac12
    \begin{vmatrix}
            1 & x & y\\
            1 & x_{3} & y_{3}\\
            1 & x_{1} & y_{1}
    \end{vmatrix}
  / \Delta = \pbrac{a_{2} + b_{2}x + c_{2}y}/ \pbrac{2 \Delta}
\end{equation*}
\begin{equation*}
  L_{3}=\dfrac{\text{Area $<P12>$}}{\text{Area $<123>$}}=\frac12
    \begin{vmatrix}
            1 & x & y\\
            1 & x_{1} & y_{1}\\
            1 & x_{2} & y_{2}
    \end{vmatrix}
  / \Delta = \pbrac{a_{3} + b_{3}x + c_{3}y}/ \pbrac{2 \Delta}
\end{equation*}
where $a_{2} = x_{3} y_{1} - x_{1} y_{3}, b_{2} = y_{3} - y_{1}, c_{2} = x_{1}
- x_{3}$ and $a_{3} = x_{1} y_{2} - x_{2} y_{1}, b_{3} = y_{1} - y_{2}, c_{3}
= x_{2} - x_{1}$.

Notice that $L_{1} + L_{2} + L_{3} =1$.

Area coordinate $L_{1}$ varies linearly from $L_{1}=0$ when $P$ lies at node $2$
or $3$ to $L_{1}=1$ when $P$ lies at node $1$ and can therefore be used directly as
the basis function for node $1$ for a three node triangle.  Thus, interpolation
over the triangle is given by
\begin{equation*}
  \fnof{u}{x,y} = \lbfn{1}{x,y}u_{1} + \lbfn{2}{x,y}u_{2} + \lbfn{3}{x,y}u_{3}
\end{equation*} %reference needed???
where $\lbfnsymb{1} = L_{1}$, $\lbfnsymb{2} = L_{2}$ and $\lbfnsymb{3} = L_{3}
= 1 - L_{1} - L_{2}$.

Six node quadratic triangular elements are constructed as shown in 
\figref{fig:6nodeqt}.
\begin{figure}[htbp] \centering
  \input{figs/fem_basis_fns/6nodeqt.pstex}
  \caption{Basis functions for a six node quadratic triangular element.}
  \label{fig:6nodeqt}
\end{figure}

%\begin{example}{Defining a triangular element mesh}
%  {Defining a triangular element mesh}
%  \begin{figure}[htbp] \centering
%    \input{figs/fem_basis_fns/egdiag5.pstex}
%    \label{fig:egdiag5}
%  \end{figure} 
%  \label{sec:Deftem} 
%\end{example}
\index{Triangular elements|)}\index{Basis functions|)}

\section{Curvilinear Coordinate Systems}
\label{sec:ccs-1.8}
\index{Curvilinear coordinate systems|(}
It is sometimes convenient to model the geometry of the region (over which a 
finite element solution is sought) using an orthogonal curvilinear coordinate 
system. A 2D circular annulus, for example, can be modelled geometrically 
using one element with cylindrical polar $\pbrac{r,\theta}$-coordinates, \eg 
the annular plate in \figref{fig:dca}a has two global nodes, the first with 
$r=r_{1}$ and the second with $r=r_{2}$.

\begin{figure}[htbp] \centering
 \input{figs/fem_basis_fns/defca.pstex}
  \caption{Defining a circular annulus with one cylindrical polar element. 
    Notice that element vertices $1$ and $2$ in $\pbrac{r,\theta}$-space or
    $\pbrac{\xione,\xitwo}$-space, as shown in (b) and (c), respectively, map onto
    the single global node $1$ in $\pbrac{x ,y}$-space in (a).  Similarly, element
    vertices $3$ and $4$ map onto global node $2$.}
  \label{fig:dca}
\end{figure}
 
Global nodes $1$ and $2$, shown in $\pbrac{x,y}$-space in \figref{fig:dca}a, each 
map to two element vertices in $\pbrac{r,\theta}$-space, as shown in
\figref{fig:dca}b, and in $\pbrac{\xione,\xitwo}$-space, as shown in
\figref{fig:dca}c. The $\pbrac{r,\theta}$ coordinates at any
$\pbrac{\xione,\xitwo}$ point are given by a bilinear interpolation of the nodal
coordinates $r_{n}$ and $\theta_{n}$ as
\begin{align*}
  r &= \dotprod{\lbfn{n}{\xione,\xitwo}}{r_{n}}\\
  \theta &= \dotprod{\lbfn{n}{\xione,\xitwo}}{\theta_{n}} 
\end{align*}
where the basis functions $\lbfn{n}{\xione,\xitwo}$ are given by \bref{eqn:2,3DE}.

%\begin{example}{Defining a bilinear mesh in cylindrical polar coordinates}
%  {Defining a bilinear mesh in cylindrical polar coordinates}
%  \label{sec:Defbm}
%\end{example} 

Three orthogonal curvilinear coordinate systems are defined here for use in
later sections.

\begin{description}
\item \textbf{Cylindrical polar\index{Curvilinear coordinate
      systems!Cylindrical polar} $\pbrac{r,\theta, z}$ :} 
  \begin{equation}
    \begin{split}
      x & = r \cos\theta \\
      y & = r \sin\theta \\
      z & = z 
    \end{split}
  \end{equation}

\item \textbf{Spherical polar\index{Curvilinear coordinate
      systems!Spherical polar} $\pbrac{r,\theta, \phi}$ :}
  \begin{equation}
    \begin{split}
      x & = r \cos\theta \cos\phi \\
      y & = r \sin\theta \cos\phi \\
      z & = r \sin\phi 
    \end{split}
  \end{equation}
  
\item \textbf{Prolate spheroidal\index{Curvilinear coordinate
      systems!Prolate spheroidal} $\pbrac{\lambda, \mu, \theta}$ :}
  \begin{equation}
    \begin{split}
      x & = d \cosh\lambda \cos\mu \\ 
      y & = d \sinh\lambda \sin\mu \cos\theta \\ 
      z & = d \sinh\lambda \sin\mu \sin\theta 
    \end{split}
  \end{equation}
\end{description}


\begin{figure}[htbp] \centering
  \input{figs/fem_basis_fns/Psc.pstex}
  \caption{Prolate spheroidal coordinates.}
  \label{fig:psc}
\end{figure}


The prolate spheroidal coordinates rae illustrated in \figref{fig:psc} and a single prolate spheroidal element is shown in \figref{fig:spse}. The
coordinates $\pbrac{\lambda, \mu, \theta}$ are all trilinear in
$\pbrac{\xione,\xitwo,\xithree}$.  Only four global nodes are required provided
the four global nodes map to eight element nodes as shown in \figref{fig:spse}.

\begin{figure}[H] \centering
  \input{figs/fem_basis_fns/spse.pstex}
  \caption{A single prolate spheroidal element, shown (a) in 
    $\pbrac{x,y,z}$-coordinates, (c) in $\pbrac{\lambda, \mu,
      \theta}$-coordinates and (d) in
    $\pbrac{\xione,\xitwo,\xithree}$-coordinates, (b) shows the orientation
    of the $\xi_{i}$-coordinates on the prolate spheroid.}
  \label{fig:spse}
\end{figure}
\index{Curvilinear coordinate systems|)}

\clearpage

\section{CMISS Examples}

\begin{enumerate}
\item  To define a 2D bilinear finite element mesh run the CMISS example number
  $111$.  The nodes should be positioned as shown in \figref{fig:example111_node}.
  After defining elements the mesh should appear like the one shown in
  \figref{fig:example111_elem}.

  \begin{figure}[htbp] \centering
    \input{figs/fem_basis_fns/egdiag1.pstex}
    \caption{Node positions for example $111$.}
    \label{fig:example111_node}
  \end{figure}

   \begin{figure}[hbtp] \centering
    \input{figs/fem_basis_fns/egdiag2.pstex}
    \caption{2D bilinear finite element mesh for example $111$.}
    \label{fig:example111_elem}
  \end{figure} 
  \label{xmp:Def2D}

\item  To refine a mesh run the CMISS example $113$. After the first refine the
  mesh should appear like the one shown in \figref{fig:example113}.  
  \begin{figure}[htbp] \centering
    \input{figs/fem_basis_fns/egdiag3.pstex}
    \caption{First refined mesh for example $113$}
    \label{fig:example113}
  \end{figure}
  \begin{figure}[htbp] \centering
    \input{figs/fem_basis_fns/egdiag4.pstex}
    \caption{Second refined mesh for example $113$}
    \label{fig:example113b}
  \end{figure}
  \label{xmp:Refining}

\item  To define a quadratic-linear element run the cmiss example $115$.
\item  To define a 3D trilinear element run CMISS example $121$.
\item  To define a 2D cubic Hermite-linear finite element mesh run 
    example $114$.

\item  To define a triangular element mesh run CMISS example $116$ (see \figref{fig:example116}).
  \begin{figure}[htbp] \centering
    \input{figs/fem_basis_fns/egdiag5.pstex}
    \label{fig:egdiag5}
    \caption{Defining a triangular mesh for example $116$}
    \label{fig:example116}
  \end{figure} 
  \label{sec:Deftem} 

\item  To define a bilinear mesh in cylindrical polar coordinates run CMISS
  example $122$.
  \label{sec:Defbm}
\end{enumerate}

%%% Local Variables: 
%%% mode: latex
%%% TeX-master: "/product/cmiss/documents/notes/fembemnotes/fembemnotes"
%%% End: 
