\subsection{Convective Transport}

For fluid problems with convective terms, additional techniques beyond the standard Galerkin formulation are often used to account for instabilities due to large, nonsymmetric gradients in the flow direction that would otherwise place difficult constraints on the discretization of the domain. Here we will discuss the Petrov-Galerkin formulation, as it is implemented in OpenCMISS. For a complete description of these stabilized methods, we refer to (Zienkiewicz, Heinrich, Chung). 

\subsubsection{Steady Advection-Diffusion}

Let us first consider linear convective transport for 1D convection-diffusion with $\phi$ transported by velocity u and diffusivity characterized by D:

\begin{equation}
%  -\frac{d}{dx}\pbrac{D\frac{d\phi}{dx}}+u\frac{d\phi}{dx}=0
  -D\pbrac{\frac{d\phi^2}{d^2x}}+u\frac{d\phi}{dx}=0
  \label{eqn:AdvectionDiffusion}
\end{equation}
Where the first term represents diffusion of $\phi$ and the second advection. The ratio of these two processes over a spatial domain can be described by the dimensionless Peclet number:
\begin{equation}
 \gamma=\frac{Lu}{D}
\end{equation}
Large values for $\gamma$ indicate that the length scale for the element, L, and the velocity, u, become much larger than the diffusion length scale, D. Numerically, large $\gamma$ values produce oscillations in the classic Galerkin formulation, as the advective terms preclude the 2nd order diffusive terms, causing problems in regions like boundary layers. It should also be noted that, while it is not as obvious in the 1D case, the directionality of the velocity vector is important in the determination of the effective Peclet number and it's effect on numerical stability.

A fairly obvious remedy for this issue is to ensure that the mesh is sufficiently refined so that values of L counteract the effect of either low diffusivity or high convection. However, for many cases of interest, it is not computationally tractable to resolve meshes so that the length scale is on the order of the diffusivity scale, or, for highly convective flows, much lower to counteract the velocity term. Therefore, a number of stabilized finite element methods have been developed, many of which deal with convective flows by adding a numerical 'balancing diffusion' to the system.

One method of adjusting the problem to reduce the element Peclet number without large mesh refinement is to introduce an artificial diffusion term $\alpha$ to the advection-diffusion equation:

\begin{equation}
%  -\frac{d}{dx}\pbrac{D\frac{d\phi}{dx}}+u\frac{d\phi}{dx}=0
%  -\pbrac{D+\frac{{\alpha}uL}{2}}\pbrac{\frac{d\phi^2}{d^2x}}+u\frac{d\phi}{dx}=0
  -\pbrac{D+{\alpha}uL}\pbrac{\frac{d\phi^2}{d^2x}}+u\frac{d\phi}{dx}=0
  \label{eqn:directupwind}
\end{equation}

Where $\alpha$ represents an adjustable factor for the artificial diffusion. It should also be noted that when $\alpha=0$, we recover the classic advection-diffusion equation (which we have described as problematic for convective flows), and larger values of $\alpha$ provide an $\emph{upwind}$ solution that adds additional diffusion dependent on the velocity and length scale. It should be noted that this is similar to backward finite difference schemes, which are also often called upwind schemes. As the artificial diffusion is also added in the direction (or along the streamline) of the velocity term, the term $\emph{streamline upwind}$ is also used to describe the scheme. The value of the $\alpha$ term may be calculated using linear elements so that the truncation error of the Taylor series is minimized, resulting in (see Heinrich, Zienkiewicz):

\begin{equation}
  \alpha= {\coth{\gamma} - \frac{1}{\gamma}}
\end{equation}

If instead of directly modifying the diffusion term in the physical system of equations, we add a consistent numerical diffusion to the weighting terms (test functions) for the Galerkin weak form of the system, we obtain a Petrov-Galerkin formulation. For the advection-diffusion example described above, the weak form after integration by parts is (Heinrich):

\begin{equation}
% \gint{0}{L}{\pbrac{D\frac{dw}{dx}\frac{{d}c}{dx}+wu\frac{{d}c}{dx}}}x+w\pbrac{-D\frac{{d}c}{dx}}=0
 \gint{0}{L}{\pbrac{D\frac{dw}{dx}\frac{{d}\phi}{dx}+wu\frac{{d}\phi}{dx}}}x=0
\end{equation}
where $w$ represents the test function as described in section (???- still to be written!). We can modify this test function so that:
\begin{equation}
 w_i = \Phi_i + \Psi_i
\end{equation}
where $\Phi_i$ represents the standard Galerkin test function and $\Psi_i$ will be our numerical diffusion test function. If we use linear elements and define $\Psi$ so that $\Psi={\alpha}L\frac{d\Phi}{dx}$, we will achieve the same effect as directly adding additional diffusion as in $\eqref{directupwind}$ since the second order shape derivatives for linear elements will be 0, giving:
\begin{equation}
% \gint{0}{L}{\pbrac{D\frac{d\Phi}{dx}\frac{{d}\phi}{dx}+{u}\pbrac{\Phi+\frac{{\alpha}L}{2}\frac{d\Phi}{dx}}\frac{{d}\phi}{dx}}}x=0
 \gint{0}{L}{\pbrac{D\frac{d\Phi}{dx}\frac{{d}\phi}{dx}+{u}\pbrac{\Phi+{\alpha}L\frac{d\Phi}{dx}}\frac{{d}\phi}{dx}}}x=0
\end{equation}
The Petrov-Galerkin formualtion is also more useful with the introduction of source terms, as it consistently weights the system with numerical diffusion, rather than simply adding to the diffusion term on the LHS (in fact we are adding a diffusive weight to the convective term in this case). Examples illustrating this can be found in (Leonard, Heinrich). For additional techniques designed to accomodate higher order shape functions, we refer to (Hughes). Also, it should be mentioned that given very high velocity gradients, as in cases of shocks, numerical oscillations still persist and the addition of discontinuity capturing schemes may be neccesary (Chung). However, this is currently not implementated in OpenCMISS.

\subsubsection{Dynamic Convective Transport}

With the introduction of unsteady terms to the system, we must also modify our Petrov-Galerkin method to accomodate for damping due to convective transport across the mesh through time. Similar to the way in which the Peclet number describes the ratio of the advective and diffusive scales, we use the Courant number, $c$, to describe how the velocity $u$ of the species in consideration advances over the element length, $L$, for a given time step $\Delta{t}$:

\begin{equation}
 c=\frac{u\Delta{t}}{L}
\end{equation}

As a solution proceeds from $t$ to $t+\Delta{t}$, sharp velocity gradients will more often fall between nodes of the mesh for increasing values of $c$. When the highest $u$ values fall between nodes, the solver must compensate for this by redistributing the mass in the system so that conservation holds. For steady-state solutions solved dynamically, this can be useful as features like shocks may be quickly damped out of the system. However, if the evolution of the solution as it progresses and these convective processes are of interest, care must be taken to refine the appropriate time-step to meet a specified value for $c$. Generally, $c\le{1}$ is often neccessary in these cases to capture flow features, even when combined with transient schemes that are unconditionally stable (e.g. Crank-Nicolson ($\theta=\frac{1}{2}$), backward implicit ($\theta=1$).) This increases the likeliehood of the mesh being able to capture complex velocity patterns during the solution but does not guaranteee it- therefore time refinement studies are often required for solutions, as spatial mesh refinement studies are for steady state solutions. 

%As $\alpha$ was used to introduce numerical diffusion to the Galerkin method above, another factor, $\beta$, will be used to add a term to the system to counteract the damping effect of the time solution. To continue this explaination, use a continuous space-time method to describe transient behavior, using a time-element with linear space, quadratic time weights. $\eqref{directupwind}$ will become:

Using a continuous space-time method and a quadratic time-element, it is possible to introduce another factor to add a term to the system to counteract the damping effect of the time solution. However, we shall describe the procedure using a discontinuous formulation, as it is currently implemented and is more efficient, as it avoids adding subsequent dimensions to the solution procedure. 

\begin{equation}
  \frac{\del\phi}{\del{t}}-\pbrac{D+{\alpha}uL}\pbrac{\frac{d\phi^2}{d^2x}}+u\frac{d\phi}{dx}+\beta{uL\Delta{t}}\frac{\del^3\phi}{\del^2{x}\del{t}}=0
\end{equation}

This will give a new test function 

%In the case of very high convective gradients like shocks, discontinuity-capturing schemes may be used but are not currently implemented in OpenCMISS.

For nonlinear problems, the Petrov-Galerkin test function may be applied directly if the convective terms are calculated explicitly. 

Even without an explicit diffusive term like in the above advection-diffusion example, such a mismatch between the length scale for a given spatial discretization and large velocity gradients over the space they describe still cause similar oscillatory problems using the classic Galerkin formulation for many fluid mechanics problems. For nonlinear equations like Burgers or Navier-Stokes, the Petrov-Galerkin scheme also becomes more complicated as it needs to be implemented as part of the nonlinear iteration. 

It should be noted that introducing an artificial diffusion term in this manner is somewhat contentious- as suppressing the oscillatory errors or 'wiggles' due to the highly restrictive effect of the classic Galerkin method on mesh design for these flows can also prevent practitioners from using this feedback to adjust their simulation and (particularly) mesh resolution. We refer the interested reader to Gresho and Sani and encourage them to keep this in mind but maintain that Petrov-Galerkin formulations are an invaluable tool for solving computationally tractable convective FEM fluid problems.


%%% Local Variables: 
%%% mode: latex
%%% TeX-master: "../../OpenCMISSNotes"
%%% End: 
