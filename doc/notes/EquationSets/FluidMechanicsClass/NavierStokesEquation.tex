\subsection{Navier-Stokes Equations} 

\subsubsection{Governing equations:}

The Navier-Stokes equations arise from applying Newton's second law to fluid motion, i.e. the temporal and spatial fluid inertia is in equilibrium with internal (volume/body)  and external (surface) forces. The reaction of surface forces can be described in terms of the fluid stress as the sum of a diffusing viscous term, plus a pressure term. A solution of the Navier-Stokes equations is called a flow field, i.e. velocity and pressure field, which is a description of the fluid at a given point in space and time.  In the common case of an incompressible Newtonian fluid, the nonlinear Navier-Stokes equations (three-dimensional, transient) can be written using 'primitive variables' (i.e. u-velocity, p-pressure) as:
\begin{equation}
  \label{eqn:NSE}
    \rho\delby{\vect{u}}{t}+\rho(\vect{u}\cdot\grad)\vect{u}=\vect{f}-\grad{p}+\mu\laplacian{\vect{u}}
\end{equation}
accompanied by the conservation of mass (incompressibility)
\begin{equation}
  \label{eqn:NSEIncompressibility}
  \diverg{\vect{u}}=0
\end{equation}
where $\vect{u}(\vect{x},t)=(u_1,u_2,u_3)^T$ is the velocity vector depending on spatial coordinates $\vect{x}=(x_1,x_2,x_3)^T$ and the time $t$, $p$ is the scalar pressure, $\vect{f}$ an applied body force, and the material parameters $\mu$ and $\rho$ are the fluid viscosity and density, respectively.The first term represents unsteady accelerative inertial contributions, the second represents the nonlinear convective acceleration terms, the $\grad{p}$ term the pressure contributions, and the last term represents viscous stresses in the system.
% incompressibility and the LBB condition
As with Stokes flow, the incompressibility condition $\diverg{\vect{u}}=0$ also creates restrictions on the formulation known as the Ladyzhenkaya, Babuska, and Brezzi(LBB) or inf-sup consistency condition. Several methods have been devised to define a pressure function that is consistent with the velocity space using primitive variables. These include mixed element methods, penalty methods, generalized petrov-galerkin methods using pressure poisson correction, operator splitting, and semi-implicit pressure correction (Chung).

Using a classic Galerkin formulation, mixed methods are perhaps conceptually the most straightforward method of satisfying LBB, in which velocity is defined over a space one order higher than pressure (e.g. quadratic elements for velocity, linear for pressure), allowing incompressibility to be satisfied. It should be noted that our use of a mixed formulation to satisfy LBB will also be reflected in the shape functions that our weak formulation depends on. For example, using 2D elements with biquadratic velocity and linear pressure, we will have 9 weight functions for each velocity component and 4 for the pressure. 
% meaning of pressure in NSE.

\subsubsection{Weak Formulation}
% Galerkin formulation
The weak form of equations \eqref{NSE} and \eqref{NSEIncompressibility} can be found by applying standard Galerkin test functions $\vect{w}$:
\begin{equation}
 \label{BasicGalerkinNSE}
  \gint{\Omega}{}{\pbrac{\rho\delby{\vect{u}}{t}
    +\rho\vect{u}\cdot\grad{\vect{u}}
    -\vect{f}
    -\mu\laplacian{\vect{u}}
    +\grad{p}}{\vect{w}}}{\Omega} = 0
\end{equation}

Integrating by parts, we will get the weak form of the system of PDEs with the associated natural boundary conditions at the boundary $\Gamma_N$:
\begin{multline}
 \label{GalerkinNSE}
  \gint{\Omega}{}{\rho\delby{\vect{u}}{t}{\vect{w}}}{\Omega}
 +\gint{\Omega}{}{\rho\vect{u}\cdot\grad\vect{u}{\vect{w}}}{\Omega}
 +\gint{\Omega}{}{p\grad{\vect{w}}}{\Omega}
 -\gint{\Omega}{}{\mu\grad{\vect{u}}\cdot\vect{\grad{w}}}{\Omega}=\\
 \gint{\Omega}{}{\vect{f}\vect{w}}{\Omega}
 +\gint{\Gamma_N}{}{\mu\dotprod{\grad\vect{u}}{\vect{n}}\vect{w}}{\Gamma_N}
 -\gint{\Gamma_N}{}{\dotprod{p}{\vect{n}}\vect{w}}{\Gamma_N}
\end{multline}

% Boundary conditions
For more extensive discussion of this procedure, along with other weak forms of the PDEs, we refer to (Gresho and Sani). From this weak form, we see natural (Neumann) boundary conditions arising as a direct result of the integration. Neumann boundary conditions will therefore consist of a pressure term and viscous stress acting normal to a given boundary. 
\begin{gather}
 \label{eqn:NSENeumannBC}  
   \fnof{\vect{q}}{\vect{x},t} = \mu\dotprod{\grad{\fnof{\vect{u}}{\vect{x},t}}}{\normal}-p\cdot{\normal}  \\
%   \fnof{\vect{q_2}}{\vect{x},t} = p\cdot{\normal} \\
   \quad \vect{x}\in\Gamma_{N}
\end{gather}
Specification of Neumann boundaries will therefore require the specification of the viscous stress and pressures across lower order element DOFs using the mixed formulation. Dirichlet boundary conditions on a boundary $\Gamma_D$ for velocity or pressure will take the form:
\begin{gather}
 \label{eqn:NSEDirichletBC} 
  \fnof{\vect{u}}{\vect{x},t} = \fnof{\vect{d}}{\vect{x},t} \quad \vect{x}\in\Gamma_{D}\\
  \fnof{p}{\vect{x},t} = \fnof{\vect{d}}{\vect{x},t} \quad \vect{x}\in\Gamma_{D}
\end{gather}
Which may be directly applied to the relevant DOFs.

\subsubsection{Tensor notation}
Assuming no forcing terms and substituting the natural boundary as defined above, equation \eqref{GalerkinNSE} in tensor notation becomes:
\begin{multline}
 \label{TensorNSE}
  \gint{\Omega}{}{\rho\dot{u}_{i}w_{i}}{\Omega}
 +\gint{\Omega}{}{{\rho}G^{jk}u_{j}\covarderiv{u_{i}}{k}w_{i}}{\Omega}
 +\gint{\Omega}{}{G^{jk}{p_{i}}\covarderiv{w_{i}}{k}}{\Omega}\\
 -\gint{\Omega}{}{{\mu}G^{jk}\covarderiv{u_{i}}{j}\covarderiv{w_{i}}{k}}{\Omega}
 -\gint{\Gamma_N}{}{{q_{i}}{w_i}}{\Gamma_N}=\vect{0}
\end{multline}
or
\begin{multline}
 \label{Tensor2NSE}
  \gint{\Omega}{}{\rho\dot{u}_{i}w_{i}}{\Omega}
 +\gint{\Omega}{}{{\rho}G^{jk}{u_j}\pbrac{\partialderiv{u_{i}}{k}-\christoffel{i}{k}{h}u_h}{w_{i}}}{\Omega}
 +\gint{\Omega}{}{G^{jk}{p_{i}}\pbrac{\partialderiv{w_{i}}{k}-\christoffel{i}{k}{h}w_h}}{\Omega}\\
 -\gint{\Omega}{}{{\mu}G^{jk}\pbrac{\partialderiv{u_{i}}{j}-\christoffel{i}{j}{h}u_h}\pbrac{\partialderiv{w_{i}}{k}-\christoffel{i}{k}{h}w_h}}{\Omega}
 -\gint{\Gamma_N}{}{{q_{i}}{w_i}}{\Gamma_N}=\vect{0}
\end{multline}


where $G^{jk}$ is the contravariant metric tensor and
$\christoffel{i}{j}{k}$ is the Christoffel symbol of the second kind for the spatial coordinates.

\subsubsection{Finite Element Formulation}

We can now discretise the domain into finite elements \ie $\Omega=\displaystyle{\bigcup_{e=1}^{E}}\Omega_{e}$ with $\Gamma=\displaystyle{\bigcup_{f=1}^{F}}\Gamma_{f}$. \Eqnref{Tensor2NSE} now becomes:

\begin{multline}
 \label{FEMNSE}
  \dsum_{e=1}^{E}\gint{\Omega}{}{\rho\dot{u}_{i}w_{i}}{\Omega}
 +\dsum_{e=1}^{E}\gint{\Omega}{}{{\rho}G^{jk}{u_j}\pbrac{\partialderiv{u_{i}}{k}-\christoffel{i}{k}{h}u_h}{w_{i}}}{\Omega}
 +\dsum_{e=1}^{E}\gint{\Omega}{}{G^{jk}{p_{i}}\pbrac{\partialderiv{w_{i}}{k}-\christoffel{i}{k}{h}w_h}}{\Omega}\\
 -\dsum_{e=1}^{E}\gint{\Omega}{}{{\mu}G^{jk}\pbrac{\partialderiv{u_{i}}{j}-\christoffel{i}{j}{h}u_h}\pbrac{\partialderiv{w_{i}}{k}-\christoffel{i}{k}{h}w_h}}{\Omega}
 -\dsum_{f=1}^{F}\gint{\Gamma_N}{}{{q_{i}}{w_i}}{\Gamma_N}=\vect{0}
\end{multline} 
or


\begin{multline}
 \label{FEMNSE2}
  \dsum_{e=1}^{E}\gint{\Omega}{}{\rho\dot{u}_{i}w_{i}}{\Omega}
 +\dsum_{e=1}^{E}\gint{\Omega}{}{{\rho}G^{jk}{u_j}\pbrac{\partialderiv{u_{i}}{k}-\christoffel{i}{k}{h}u_h}{w_{i}}}{\Omega}
 +\dsum_{e=1}^{E}\gint{\Omega}{}{G^{jk}{p_{i}}\pbrac{\partialderiv{w_{i}}{k}-\christoffel{i}{k}{h}w_h}}{\Omega}\\
 -\dsum_{e=1}^{E}\gint{\Omega}{}{{\mu}G^{jk}\pbrac{\partialderiv{u_{i}}{j}-\christoffel{i}{j}{h}u_h}\pbrac{\partialderiv{w_{i}}{k}-\christoffel{i}{k}{h}w_h}}{\Omega}=\\
 \dsum_{f=1}^{F}\gint{\Gamma_N}{}{{\mu}G^{jk}\pbrac{\partialderiv{u_{i}}{k}-\christoffel{i}{k}{h}u_h}{n_{i}}{w_{i}}}{\Gamma_N}
-\dsum_{f=1}^{F}\gint{\Gamma_N}{}{G^{jk}{p}{n_i}{w_i}}{\Gamma_N}
\end{multline} 


We will assume that the dependent variables $\vect{u}$ and $p$ can be interpolated separately in space and time. Here we must also be careful to note again the discrepancy between the functional spaces for velocity and pressure using a mixed formulation to satisfy the LBB consistency requirement. We will therefore define two separate weighting functions: for the velocity space on $\Omega$, the test function will be $\psi_i$ and for the pressure space, $\phi_i$, giving:

\begin{gather}
  \fnof{\vect{u}}{\vect{x},t}=\gbfn{n}{}{\vect{x}}\fnof{\nodept{\vect{u}}{n}}{t}\\
  \fnof{p}{\vect{x},t}=\phi_{n}({\vect{x}})\fnof{\nodept{p}{n}}{t}
\end{gather}

In standard interpolation notation within an element, we transform from $\vect{x}$ to $\vect{\xi}$:

\begin{gather}
  \fnof{u_{i}}{\vect{\xi},t}=\idxgbfn{i}{n}{\beta}{\vect{\xi}}
  \fnof{\idxnodedof{u}{i}{n}{\beta}}{t}\idxgsf{i}{n}{\beta}\\
  \fnof{p_{}}{\vect{\xi},t}=\phi_{in}^{\beta}({\vect{\xi}})
  \fnof{\idxnodedof{p}{}{n}{\beta}}{t}\idxgsf{i}{n}{\beta}
\end{gather}

where $\fnof{\idxnodedof{u}{i}{n}{\beta}}{t}$ are the time varying nodal degrees-of-freedom for velocity component $i$, node $n$, global derivative $\beta$, $\idxgbfn{i}{n}{\beta}{\vect{\xi}}$ are the corresponding velocity basis functions and $\idxgsf{i}{n}{\beta}$ are the scale factors. The scalar pressure DOFs, $\fnof{\idxnodedof{p}{}{n}{\beta}}{t}$ are interpolated similarly.

For the natural boundary, we can separate $q_i$ into its component velocity and pressure terms as noted in \eqref{NSENeumannBC} and shown in \eqref{FEMNSE2}. For our current treatment, we will also assume the values of viscosity $\mu$ and density $\rho$ are constant. These can be interpolated:

\begin{equation}
  \begin{split}
    \fnof{q_{i}}{\vect{\xi},t} &= \idxgbfn{i}{o}{\gamma}{\vect{\xi}}
      \fnof{\idxnodedof{q}{i}{o}{\gamma}}{t}\idxgsf{i}{o}{\gamma} \\
    \fnof{\mu}{\vect{\xi}} &=\gbfn{r}{\delta}{\vect{\xi}}
    \nodedof{\mu}{r}{\delta}\gsf{r}{\delta} \\
    \fnof{\rho}{\vect{\xi}} &=\gbfn{r}{\delta}{\vect{\xi}}\nodedof{\rho}{r}{\delta}
    \gsf{r}{\delta} \\
  \end{split}
\end{equation}

 Using the spatial weighting functions for a Galerkin finite element formulation:
\begin{equation}
  \fnof{w_{i}}{\vect{\xi}}=\idxgbfn{i}{m}{\alpha}{\vect{\xi}}\idxgsf{i}{m}{\alpha}
\end{equation}


\subsubsection{Spatial Integration}

Using standard integration notation, we can rewrite our Galerkin FEM formulation from \eqref{NSEFEM2}:


\begin{multline}
 \label{NSEFEM3}
%time dependence
  \dsum_{e=1}^{E}\gint{\vect{0}}{\vect{1}}{\fnof{\rho}{\vect{\xi}}\delby{\idxgbfn{i}{n}{\beta}{\vect{\xi}}
      \fnof{\idxnodedof{u}{i}{n}{\beta}}{t}\idxgsf{i}{n}{\beta}}{t}
    \idxgbfn{i}{m}{\alpha}{\vect{\xi}}\idxgsf{i}{m}{\alpha}
    \abs{\fnof{\matr{J}}{\vect{\xi}}}}{\vect{\xi}}\\
%convective term
  - \dsum_{e=1}^{E}\dintl{\vect{0}}{\vect{1}}{\fnof{\rho}{\vect{\xi}}}G^{jk}\idxgbfn{j}{n}{\beta}{\vect{\xi}}
  \fnof{\idxnodedof{u}{j}{n}{\beta}}{t}\idxgsf{i}{n}{\beta}\\
  \pbrac{\delby{\idxgbfn{i}{n}{\beta}{\vect{\xi}}\fnof{\idxnodedof{u}{i}{n}{\beta}}{t}
      \idxgsf{i}{n}{\beta}}{x^{k}}-\christoffel{i}{k}{h}\idxgbfn{h}{m}{\alpha}{\vect{\xi}}
    \fnof{\idxnodedof{u}{h}{n}{\beta}}{t}\idxgsf{h}{m}{\alpha}}
  \idxgbfn{i}{m}{\alpha}{\vect{\xi}}\idxgsf{i}{m}{\alpha}
  \abs{\fnof{\matr{J}}{\vect{\xi}}}d\vect{\xi}\\
%pressure
  +\dsum_{e=1}^{E}\dintl{\vect{0}}{\vect{1}}G^{jk}\phi_{in}^{\beta}({\vect{\xi}})
  \fnof{\idxnodedof{p}{}{n}{\beta}}{t}\idxgsf{i}{n}{\beta}\\
  \pbrac{\delby{\psi_{im}^{\alpha}({\vect{\xi}})\idxgsf{i}{m}{\alpha}}{x^{k}}-
  \christoffel{i}{k}{h}\psi_{im}^{\alpha}({\vect{\xi}})\idxgsf{h}{m}{\alpha}
    }\abs{\fnof{\matr{J}}{\vect{\xi}}}d\vect{\xi}\\
%viscous stress
  -\dsum_{e=1}^{E}\dintl{\vect{0}}{\vect{1}}{\fnof{\mu}{\vect{\xi}}}G^{jk}\pbrac{\delby{\idxgbfn{i}{n}{\beta}{\vect{\xi}}
     \fnof{\idxnodedof{u}{i}{n}{\beta}}{t}\idxgsf{i}{n}{\beta}}{x^{j}}-
      \christoffel{i}{j}{h}\idxgbfn{h}{n}{\beta}{\vect{\xi}}
      \fnof{\idxnodedof{u}{h}{n}{\beta}}{t}\idxgsf{h}{n}{\beta}} \\
    \pbrac{\delby{\idxgbfn{i}{m}{\alpha}{\vect{\xi}}\idxgsf{i}{m}{\alpha}}{x^{k}}-
      \christoffel{i}{k}{l}\idxgbfn{l}{m}{\alpha}{\vect{\xi}}
      \fnof{\idxnodedof{u}{h}{n}{\beta}}{t}
      \idxgsf{l}{m}{\alpha}}\abs{\fnof{\matr{J}}{\vect{\xi}}}d\vect{\xi}\\
%boundary terms
  =\dsum_{f=1}^{F}\gint{\vect{0}}{\vect{1}}{\idxgbfn{i}{o}{\gamma}{\vect{\xi}}
    \fnof{\idxnodedof{q}{i}{o}{\gamma}}{t}\idxgsf{i}{o}{\gamma}
    \idxgbfn{i}{m}{\alpha}{\vect{\xi}}\idxgsf{i}{m}{\alpha}
    \abs{\fnof{\matr{J}}{\vect{\xi}}}}{\vect{\xi}}
% note: alternate symbol for normal vector needed if want to expand boundary terms
%=\pbrac{\delby{\idxgbfn{i}{n}{\beta}{\vect{\xi}}\fnof{\idxnodedof{u}{i}{n}{\beta}}{t}
%      \idxgsf{i}{n}{\beta}}{x^{k}}-\christoffel{i}{k}{h}\idxgbfn{h}{m}{\alpha}{\vect{\xi}}
%    \fnof{\idxnodedof{u}{h}{n}{\beta}}{t}\idxgsf{h}{m}{\alpha}}\\
%  \idxgbfn{i}{m}{\alpha}{\vect{\xi}}\idxgsf{i}{m}{\alpha}
%  \abs{\fnof{\matr{J}}{\vect{\xi}}}d\vect{\xi}\\
\end{multline}

Where $\fnof{\matr{J}}{\vect{\xi}}$ represents the jacobian matrix. If we assume a rectangular cartesian coordinate system, this simplifies significantly, as the contravariant $G^{jk}$ will become $\delta^{jk}$ and the Christoffel symbols $\christoffel{i}{j}{k}$ will all be zero. This gives:

\begin{multline}
 \label{NSEFEM3}
%time dependence
  \dsum_{e=1}^{E}\gint{\vect{0}}{\vect{1}}{\fnof{\rho}{\vect{\xi}}\delby{(\idxgbfn{i}{n}{\beta}{\vect{\xi}}
      \fnof{\idxnodedof{u}{i}{n}{\beta}}{t}\idxgsf{i}{n}{\beta})}{t}
    \idxgbfn{i}{m}{\alpha}{\vect{\xi}}\idxgsf{i}{m}{\alpha}
    \abs{\fnof{\matr{J}}{\vect{\xi}}}}{\vect{\xi}}\\
%convective term
  - \dsum_{e=1}^{E}\dintl{\vect{0}}{\vect{1}}{\fnof{\rho}{\vect{\xi}}}\delta^{jk}\idxgbfn{j}{n}{\beta}{\vect{\xi}}
  \fnof{\idxnodedof{u}{j}{n}{\beta}}{t}\idxgsf{i}{n}{\beta}
  \pbrac{\delby{\idxgbfn{i}{n}{\beta}{\vect{\xi}}\fnof{\idxnodedof{u}{i}{n}{\beta}}{t}
   \idxgsf{i}{n}{\beta}}{x^{k}}}\\
  \idxgbfn{i}{m}{\alpha}{\vect{\xi}}\idxgsf{i}{m}{\alpha}
  \abs{\fnof{\matr{J}}{\vect{\xi}}}d\vect{\xi}\\
%pressure
  +\dsum_{e=1}^{E}\dintl{\vect{0}}{\vect{1}}\delta^{jk}\phi_{in}^{\beta}({\vect{\xi}})
  \fnof{\idxnodedof{p}{}{n}{\beta}}{t}\idxgsf{i}{n}{\beta}
  \pbrac{\delby{\psi_{im}^{\alpha}({\vect{\xi}})\idxgsf{i}{m}{\alpha}}{x^{k}}}\abs{\fnof{\matr{J}}{\vect{\xi}}}d\vect{\xi}\\
%viscous stress
  -\dsum_{e=1}^{E}\dintl{\vect{0}}{\vect{1}}{\fnof{\mu}{\vect{\xi}}}\delta^{jk}\pbrac{\delby{\idxgbfn{i}{n}{\beta}{\vect{\xi}}
     \fnof{\idxnodedof{u}{i}{n}{\beta}}{t}\idxgsf{i}{n}{\beta}}{x^{j}}}
    \pbrac{\delby{\idxgbfn{i}{m}{\alpha}{\vect{\xi}}\idxgsf{i}{m}{\alpha}}{x^{k}}}
    \abs{\fnof{\matr{J}}{\vect{\xi}}}d\vect{\xi}\\
%boundary terms
  =\dsum_{f=1}^{F}\gint{\vect{0}}{\vect{1}}{\idxgbfn{i}{o}{\gamma}{\vect{\xi}}
    \fnof{\idxnodedof{q}{i}{o}{\gamma}}{t}\idxgsf{i}{o}{\gamma}
    \idxgbfn{i}{m}{\alpha}{\vect{\xi}}\idxgsf{i}{m}{\alpha}
    \abs{\fnof{\matr{J}}{\vect{\xi}}}}{\vect{\xi}}
\end{multline} 

or, rearranged to pull constants out of the integrals:

\begin{multline}
 \label{NSEFEM4}
%time dependence
  \dsum_{e=1}^{E}{\idxnodedof{\dot{u}}{i}{n}{\beta}}(t){\idxgsf{i}{n}{\beta}}{\idxgsf{i}{m}{\alpha}}
  \gint{\vect{0}}{\vect{1}}{\fnof{\rho}{\vect{\xi}}{\idxgbfn{i}{n}{\beta}{\vect{\xi}}\idxgbfn{i}{m}{\alpha}{\vect{\xi}}
  \abs{\fnof{\matr{J}}{\vect{\xi}}}}}{\vect{\xi}}\\
 %convective term
  -\dsum_{e=1}^{E}{\idxnodedof{u}{j}{n}{\beta}}(t){\idxgsf{i}{n}{\beta}}^2{\idxgsf{i}{m}{\alpha}}
   \dintl{\vect{0}}{\vect{1}}\delta^{jk}{\fnof{\rho}{\vect{\xi}}\idxgbfn{j}{n}{\beta}{\vect{\xi}}
   \idxgbfn{i}{m}{\alpha}{\vect{\xi}}
     \pbrac{\delby{\idxgbfn{i}{n}{\beta}{\vect{\xi}}}{x^{k}}}
  \abs{\fnof{\matr{J}}{\vect{\xi}}}}d{\vect{\xi}}\\
 %pressure
    +\dsum_{e=1}^{E}{\fnof{\idxnodedof{p}{}{n}{\beta}}{t}\idxgsf{i}{n}{\beta}\idxgsf{i}{m}{\alpha}}
    \dintl{\vect{0}}{\vect{1}}\delta^{jk}\phi_{in}^{\beta}({\vect{\xi}})
    \pbrac{\delby{\psi_{im}^{\alpha}({\vect{\xi}})}{x^{k}}}\abs{\fnof{\matr{J}}{\vect{\xi}}}d\vect{\xi}\\
 % %viscous stress
    -\dsum_{e=1}^{E}\fnof{\idxnodedof{u}{i}{n}{\beta}}{t}\idxgsf{i}{n}{\beta}\idxgsf{i}{m}{\alpha}
    \dintl{\vect{0}}{\vect{1}}{\delta^{jk}\fnof{\mu}{\vect{\xi}}}\pbrac{\delby{\idxgbfn{i}{n}{\beta}{\vect{\xi}}}{x^{j}}}
      \pbrac{\delby{\idxgbfn{i}{m}{\alpha}{\vect{\xi}}}{x^{k}}}
      \abs{\fnof{\matr{J}}{\vect{\xi}}}d{\vect{\xi}}\\
% %boundary terms
  =\dsum_{f=1}^{F}\fnof{\idxnodedof{q}{i}{o}{\gamma}}{t}\idxgsf{i}{m}{\alpha}\idxgsf{i}{o}{\gamma}
   \gint{\vect{0}}{\vect{1}}{\gbfn{m}{\alpha}{\vect{\xi}}\gbfn{o}{\gamma}{\vect{\xi}}
    \abs{\fnof{\matr{J}}{\vect{\xi}}}}{\vect{\xi}}
\end{multline}


\subsubsection{General form}

We now seek to assemble this into the corresponding general form for dynamic equations, as outlined in section 2.3.2:

\begin{equation}
  \matr{M}\fnof{\ddot{\vect{d}}}{t}+\matr{C}\fnof{\dot{\vect{d}}}{t}+\matr{K}\fnof{\vect{d}}{t}+
  \fnof{\vect{g}}{\fnof{\vect{d}}{t}}+\fnof{\vect{f}}{t}=\vect{0}
  \label{eqn:NSEGeneralDynamic}
\end{equation}
where $\dot{\vect{d}}(t)$ and $\ddot{\vect{d}}(t)$ represent the first and second derivatives (respectively) of the degrees of freedom vector $\vect{d}(t)$, which consists of the dependent variables $\vect{u}(\vect{x},t)$ and $p(x,t)$. $\matr{M}$ is the mass matrix, which provides the shape function based weights and $\matr{C}$ is the transient damping matrix. $\matr{K}$ represents the stiffness matrix, which will contain the linear parts of the operator. $\fnof{\vect{g}}{\fnof{\vect{u}}{t}}$ is the nonlinear vector function that will be used to represent the convective term and $\fnof{\vect{f}}{t}$ the forcing vector.

 We will assume cartesian coordinates $\vect{x}=\bbrac{x,y,z}$ and denote the corresponding velocity components $\vect{u}=\bbrac{u,v,w}$, with $N$ representing the number of velocity DOFs and $M$ the number of pressure DOFs. The DOFs vector $d$ then becomes:
\begin{equation}
  \vect{d}=[u_1u_2...u_Nv_1v_2...v_Nw_1w_2...w_Np_1p_2...p_M]^T=
  \begin{bmatrix}\vect{u}\\\vect{v}\\\vect{w}\\\vect{p}\end{bmatrix}
\end{equation}
Where $u$ has been used instead of $u(t)$ for simplicity. For further illustration, consider a simple mesh consisting of two adjacent quadrilateral elements using 2D biquadratic velocity (9 nodes per element), bilinear pressure (4 nodes per element) interpolation. The mesh will consist of 15 DOFs for the velocity and 6 for pressure, as the adjoining nodes between elements are shared. The DOFs vector $\vect{d}$ will be assembled:

\begin{equation}
  \vect{d}_{example}=[u_{1}...u_{15}v_{1}...v_{15}p_1...p_{6}]^T=
  \begin{bmatrix}\vect{u}\\\vect{v}\\\vect{p}\end{bmatrix}
\end{equation}

All the components of \eqref{eqn:NSEGeneralDynamic} will similarly depend on the number of dimensions,$N_{dim}$, and the size of the matrices will be: $(N{\cdot}{N_{dim}}+M)^{N_{dim}}$. For the simple example used above, the size of the $d_{example}$ vector would be 36 and the matrices would be 36 x 36. 

Returning to the general case, the mass matrix $\matr{M}$ will take the form:

\begin{equation}
[M^{\alpha\beta}_{mn}]=\gint{\vect{0}}{\vect{1}}{\idxgbfn{i}{n}{\beta}{\vect{\xi}}\idxgbfn{i}{m}{\alpha}{\vect{\xi}}
    \abs{\fnof{\matr{J}}{\vect{\xi}}}}{\vect{\xi}}
\end{equation}
Without using mass-lumping, the damping matrix $\matr{C}$:
\begin{equation}
  [C^{\alpha\beta}_{mn}]={\idxgsf{i}{n}{\beta}}{\idxgsf{i}{m}{\alpha}}}
  \gint{\vect{0}}{\vect{1}}{\fnof{\rho}{\vect{\xi}}{\idxgbfn{i}{n}{\beta}{\vect{\xi}}\idxgbfn{i}{m}{\alpha}{\vect{\xi}}
    \abs{\fnof{\matr{J}}{\vect{\xi}}}}}{\vect{\xi}}
\end{equation}

The element stiffness matrix $\matr{K}$ will contain the linear components of the system. For the classic Galerkin formulation this includes the viscous term with the laplacian operator and the pressure gradient term. As the velocity and pressure DOFs are both included in $d$, $\matr{K}$ will be assembled so that the corresponding operators are applied to the DOFs discontinuously. 

\begin{equation}
 [K^{\alpha\beta}_{mn}]=
 \begin{cases}
   [A^{\alpha\beta}_{mn}] =  {\idxgsf{i}{n}{\beta}{\idxgsf{i}{m}{\alpha}}}
    \dintl{\vect{0}}{\vect{1}}{\delta^{jk}\fnof{\mu}{\vect{\xi}}}\pbrac{\delby{\idxgbfn{i}{n}{\beta}{\vect{\xi}}}{x^{j}}}
       \pbrac{\delby{\idxgbfn{i}{m}{\alpha}{\vect{\xi}}}{x^{k}}}
       \abs{\fnof{\matr{J}}{\vect{\xi}}}d{\vect{\xi}}\\
     \qquad\qquad\qquad\qquad\qquad\qquad\qquad\qquad\qquad\qquad\qquad\qquad \text{for $n\le{N{\cdot}{N_{dim}}}$},\\

   [B^{\alpha\beta}_{mn}] = \idxgsf{i}{n}{\beta}\idxgsf{i}{m}{\alpha}}
     \dintl{\vect{0}}{\vect{1}}\delta^{jk}\phi_{in}^{\beta}({\vect{\xi}})
     \pbrac{\delby{\psi_{im}^{\alpha}({\vect{\xi}})}{x^{k}}}\abs{\fnof{\matr{J}}{\vect{\xi}}}d\vect{\xi}\\
    \qquad\qquad\qquad\qquad\qquad\qquad\qquad\qquad\qquad\qquad\qquad\qquad \text{for $n>{N{\cdot}{N_{dim}}}$}.
 \end{cases}
\end{equation}

To attain a square matrix of the required size, we must also apply the transpose of the gradient terms acting on the pressure DOFs. For example, in our 2D example system, $\matr{K}$ will take the form: 
\begin{equation}
  \matr{K}=
   \begin{bmatrix}
     \matr{A} & \matr{0} & \matr{B}_x\\
     \matr{0} & \matr{A} & \matr{B}_y\\
     -\matr{B}^T_{x}& -\matr{B}^T_{y} & \matr{0}
   \end{bmatrix}
\end{equation}

The nonlinear vector, $\vect{g}(t)$ will provide the convective operators and for the standard Galerkin formulation:

\begin{equation}
 [g^{\alpha\beta}_{mn}]=
 {\idxgsf{i}{n}{\beta}}^2{\idxgsf{i}{m}{\alpha}}
   \dintl{\vect{0}}{\vect{1}}\delta^{jk}{\fnof{\rho}{\vect{\xi}}\idxgbfn{j}{n}{\beta}{\vect{\xi}}
   \idxgbfn{i}{m}{\alpha}{\vect{\xi}}
     \pbrac{\delby{\idxgbfn{i}{n}{\beta}{\vect{\xi}}}{x^{k}}}
  \abs{\fnof{\matr{J}}{\vect{\xi}}}}d{\vect{\xi}}
\end{equation}

\subsubsection{Streamline Upwind/Petrov-Galerkin (SUPG)}

For convection dominated flows, it is often useful to modify the Galerkin formulation account for numerical problems associated with large asymmetric  velocity gradients. We describe augmenting the Galerkin Navier-Stokes formulation to form a Petrov-Galerkin form, which uses a different test function for the convective term to stabilize the algorithm with a balancing diffusive term. Please refer to section ?? for more details. 

For the Navier-Stokes equations (and most nonlinear problems), the use of a Petrov-Galerkin formulation becomes much more difficult than for linear cases like advection-diffusion, as consistent weighting can cause instabilities when used with additional terms like body forces (Heinrich). These issues can be overcome with more complex Petrov-Galerkin formulations, as those derived by Hughes and colleagues in the 1980s. A useful alternative, though perhaps somewhat less rigorous, route is to apply SUPG weights to the convective operator in elements as a function of each element's Reynolds number (referred to subsequently as the cell Reynolds number). This can provide the desired stabilizing effect in the direction of large velocity gradients but can be easily implemented and is practically useful.    

 Assuming cartesian coordinates and applying Petrov-Galerkin weights to the convective term, the finite element formulation described in equation \eqref{eq:NSEFEM2} can be written:

\begin{multline}
 \label{NSEPGFEM1}
  \dsum_{e=1}^{E}\gint{\Omega}{}{\rho\dot{u}_{i}w_{i}}{\Omega}
 +\dsum_{e=1}^{E}\gint{\Omega}{}{{\rho}{\delta}^{jk}{u_j}\pbrac{\partialderiv{u_{i}}{k}}({w_{i}+\Psi_{i}})}{\Omega}
 +\dsum_{e=1}^{E}\gint{\Omega}{}{{\delta}^{jk}{p_{i}}{\partialderiv{w_{i}}{k}}}{\Omega}\\
 -\dsum_{e=1}^{E}\gint{\Omega}{}{{\mu}{\delta}^{jk}\pbrac{\partialderiv{u_{i}}{j}}\pbrac{\partialderiv{w_{i}}{k}}}{\Omega}=
 \dsum_{f=1}^{F}\gint{\Gamma_N}{}{{\mu}{\delta}^{jk}\pbrac{\partialderiv{u_{i}}{k}}{n_{i}}{w_{i}}}{\Gamma_N}
-\dsum_{f=1}^{F}\gint{\Gamma_N}{}{{\delta}^{jk}{p}{n_i}{w_i}}{\Gamma_N}
\end{multline}

where $w_i$ will be the classic Galerkin weight and $\Psi_i$ will be the Petrov-Galerkin for the Navier-Stokes equations. We must also define the Peclet number, $\gamma$, for the Navier-Stokes equations to describe the ratio of the convective:viscous operators. Here, the effective Peclet number will be based on the cell Reynolds number:

\begin{equation}
 Re = \frac{\rho\bar{U}L}{\mu}  =\frac{\bar{U}L}{\nu}
\end{equation}
where $\bar{U}$ is the characteristic velocity and $\nu=\frac{\mu}{\rho}$ is the kinematic viscosity. $L$ will be the characteristic length scale for a given element. The effective Peclet number $\gamma$ will be:

\begin{equation}
  \gamma= \frac{Re}{2}  = \frac{\bar{u}L}{2\nu}
\end{equation}

We will retain the same value for $\alpha={\coth{\frac{\gamma}{2}} - \frac{2}{\gamma}}$ as found for the linearized advection-diffusion equation. 

\begin{equation}
 \label{PGTest}
  \Psi_i = \pbrac{\frac{\alpha{L}}{2}{\frac{u_j}{\abs{u_j}}}}{\partialderiv{w_{i}}{k}}
\end{equation},


Equation \eqref{NSEPGFEM1} then becomes after integration and simplification:


\begin{multline}
 \label{NSEFEM4}
%time dependence
  \dsum_{e=1}^{E}{\idxnodedof{\dot{u}}{i}{n}{\beta}}(t){\idxgsf{i}{n}{\beta}}{\idxgsf{i}{m}{\alpha}}
  \gint{\vect{0}}{\vect{1}}{\fnof{\rho}{\vect{\xi}}{\idxgbfn{i}{n}{\beta}{\vect{\xi}}\idxgbfn{i}{m}{\alpha}{\vect{\xi}}
  \abs{\fnof{\matr{J}}{\vect{\xi}}}}}{\vect{\xi}}\\
 %convective term
  -\dsum_{e=1}^{E}{\idxnodedof{u}{j}{n}{\beta}}(t){\idxgsf{i}{n}{\beta}}^2{\idxgsf{i}{m}{\alpha}}
   \dintl{\vect{0}}{\vect{1}}\delta^{jk}{\fnof{\rho}{\vect{\xi}}\idxgbfn{j}{n}{\beta}{\vect{\xi}}
    \idxgbfn{i}{m}{\alpha}{\vect{\xi}}
     \pbrac{\delby{\idxgbfn{i}{n}{\beta}{\vect{\xi}}}{x^{k}}}
  \abs{\fnof{\matr{J}}{\vect{\xi}}}}d{\vect{\xi}}\\
 %P-G term
  -\dsum_{e=1}^{E}{\idxnodedof{u}{j}{n}{\beta}}(t){\idxgsf{i}{n}{\beta}}^2{\idxgsf{i}{m}{\alpha}}
   \dintl{\vect{0}}{\vect{1}}\delta^{jk}{\fnof{\rho}{\vect{\xi}}\idxgbfn{j}{n}{\beta}{\vect{\xi}}\\
    \pbrac{\frac{(\alpha){L(\vect{\xi})}}{2}{\frac{u_j}{\abs{u_j}}}}\idxgbfn{i}{m}{\alpha}{\vect{\xi}}
     \pbrac{\delby{\idxgbfn{i}{n}{\beta}{\vect{\xi}}}{x^{k}}}
  \abs{\fnof{\matr{J}}{\vect{\xi}}}}d{\vect{\xi}}\\
 %pressure
    +\dsum_{e=1}^{E}{\fnof{\idxnodedof{p}{}{n}{\beta}}{t}\idxgsf{i}{n}{\beta}\idxgsf{i}{m}{\alpha}}
    \dintl{\vect{0}}{\vect{1}}\delta^{jk}\phi_{in}^{\beta}({\vect{\xi}})
    \pbrac{\delby{\psi_{im}^{\alpha}({\vect{\xi}})}{x^{k}}}\abs{\fnof{\matr{J}}{\vect{\xi}}}d\vect{\xi}\\
 % %viscous stress
    -\dsum_{e=1}^{E}\fnof{\idxnodedof{u}{i}{n}{\beta}}{t}\idxgsf{i}{n}{\beta}\idxgsf{i}{m}{\alpha}
    \dintl{\vect{0}}{\vect{1}}{\delta^{jk}\fnof{\mu}{\vect{\xi}}}\pbrac{\delby{\idxgbfn{i}{n}{\beta}{\vect{\xi}}}{x^{j}}}
      \pbrac{\delby{\idxgbfn{i}{m}{\alpha}{\vect{\xi}}}{x^{k}}}
      \abs{\fnof{\matr{J}}{\vect{\xi}}}d{\vect{\xi}}\\
% %boundary terms
  =\dsum_{f=1}^{F}\fnof{\idxnodedof{q}{i}{o}{\gamma}}{t}\idxgsf{i}{m}{\alpha}\idxgsf{i}{o}{\gamma}
   \gint{\vect{0}}{\vect{1}}{\gbfn{m}{\alpha}{\vect{\xi}}\gbfn{o}{\gamma}{\vect{\xi}}
    \abs{\fnof{\matr{J}}{\vect{\xi}}}}{\vect{\xi}}
\end{multline}

Notice the integration by parts is done only to the standard Galerkin weights- this is because the artificial diffusion added by the Petrov-Galerkin terms should only contribute within the elements

The use of the Petrov-Galerkin formulation in this way allows for the stabilization of moderately convective problems. However, it also results in nonsymmetric mass matrices with the additional term, making classical mass-lumping much more difficult. As a result, the use of explicit methods also becomes more difficult for time-dependent problems. 

%ALE













Here $N_{u}$ represents the total $u$ velocity nodes for a given dimension and $N_p$ the
total number of $p$ pressure nodes. Our finite element formulation for
the mixed formulation becomes:

\begin{multline}
 \label{FEMNSE}
  \dsum_{i=1}^{N_u}\gint{\Omega}{}{\rho\dot{u}_{i}w_{i}}{\Omega}
 +\dsum_{i=1}^{E}\gint{\Omega}{}{{\rho}G^{jk}{u_j}\pbrac{\partialderiv{u_{i}}{k}-\christoffel{i}{k}{h}u_h}{w_{i}}}{\Omega}
 +\dsum_{i=1}^{E}\gint{\Omega}{}{G^{jk}{p_{i}}\pbrac{\partialderiv{w_{i}}{k}-\christoffel{i}{k}{h}w_h}}{\Omega}\\
 -\dsum_{i=1}^{E}\gint{\Omega}{}{{\mu}G^{jk}\pbrac{\partialderiv{u_{i}}{j}-\christoffel{i}{j}{h}u_h}\pbrac{\partialderiv{w_{i}}{k}-\christoffel{i}{k}{h}w_h}}{\Omega}
 -\dsum_{f=1}^{F}\gint{\Gamma_N}{}{{q_{i}}{w_i}}{\Gamma_N}=\vect{0}
\end{multline}



% convective Petrov-Galerkin formulation
For convective flows, we augment the convective term of \eqref{GalerkinNSE} with a Petrov-Galerkin test function of the form $\vect{\Phi}=\vect{w}+\vect{\Psi}$, where $\vect{w}$ represent standard Galerkin weights as used in \eqref{GalerkinNSE} and $\vect{\Psi}$ takes the form:
\begin{equation}
 \label{PetrovGalerkinTest}
  \vect{\Psi}=\frac{\alpha{L}}{2}
\end{equation}


%ALE form
Whereas \eqnref{eqn:NavierStokesequation1} has been formulated in Eulerian form, moving domain approaches often require the ALE modification taking an additional term into account, depending on the fluid density $\rho$ and a correction velocity $\vect{u}^*$ which yields to:
\begin{equation}
    \rho((\vect{u}-\vect{u}^*)\cdot\grad)\vect{u}=\vect{f}-\grad{p}+\mu\laplacian{\vect{u}}
  \label{eqn:NavierStokesequationALE}
\end{equation}
So far, the nonlinear term in \eqnref{eqn:NavierStokesequation1} represents the fluid spatial acceleration only. \eqnref{eqn:NavierStokesequation2} now also takes the dynamic inertia terms into account
\begin{equation}
    \rho\delby{\vect{u}}{t}+ \rho((\vect{u}-\vect{u}^*)\cdot\grad)\vect{u}=\vect{f}-\grad{p}+\mu\laplacian{\vect{u}}
  \label{eqn:NavierStokesequation2}
\end{equation}
which gives us the complete Navier-Stokes equations in ALE formulation.
The following section, however, describes the reordered quasi-static formulation of  \eqnref{eqn:NavierStokesequationALE}:
\begin{equation}
\rho((\vect{u}-\vect{u}^*)\cdot\grad)\vect{u}-\mu\laplacian{\vect{u}}+\grad{p}=\vect{f}
%     -\grad{p}+\mu\laplacian{\vect{u}}-\rho(\vect{u}^*\cdot\grad)\vect{u}=\vect{f}
  \label{eqn:NavierStokesequationALE2}
\end{equation}

%\subsubsection{Weak formulation:}

The corresponding weak form of the equation system consisting of \eqnref{eqn:NavierStokesequation1} and \eqnref{eqn:NavierStokesmasequation} can be written in the general dynamic form (see section 2.3.2)
\begin{equation}
  \matr{M}\fnof{\ddot{\vect{u}}}{t}+\matr{C}\fnof{\dot{\vect{u}}}{t}+\matr{K}\fnof{\vect{u}}{t}+
  \fnof{\vect{g}}{\fnof{\vect{u}}{t}}+\fnof{\vect{f}}{t}=\vect{0}
  \label{eqn:generaldynamicnonlinear}
\end{equation}
where u(t) is the dependent variables vector $\vect{u}(\vect{x},t)$ and $p$ for the degrees of freedom. $\matr{M}$ is the mass matrix, which provides the shape function based weights, $\matr{C}$ is the transient damping matrix (which we will discuss further below). $\matr{K}$ represents the stiffness matrix, which will contain the linear parts of the operator, including the viscous terms, the conservation of mass terms, and pressure terms. $\fnof{\vect{g}}{\fnof{\vect{u}}{t}}$ is the nonlinear vector function for the convective terms and $\fnof{\vect{f}}{t}$ the forcing vector. 

%ALE

The corresponding weak form of the equation system consisting of \eqnref{eqn:NavierStokesequation1} and \eqnref{eqn:NavierStokesmasequation} can be written as:
\begin{equation}
  \gint{\Omega}{}{\rho(\vect{u}\cdot\grad)\vect{u}\vect{v} }{\Omega}
  -\gint{\Omega}{}{\rho(\vect{u}^*\cdot\grad)\vect{u}\vect{v} }{\Omega}
  +\gint{\Omega}{}{\mu\laplacian{\vect{u}}\vect{v}}{\Omega}
  -\gint{\Omega}{}{\grad{p}\vect{v}}{\Omega}
  +\gint{\Omega}{}{\grad\cdot \vect{u}q}{\Omega}=
  \gint{\Omega}{}{\vect{f}\vect{v}}{\Omega}  
  \label{eqn:NavierStokesweakform}
\end{equation}
The general form for this kind of equation system is
\begin{equation}
  \matr{K}{\vect{\hat{u}}}+
  \fnof{\vect{\hat{g}}}{{\vect{{u}}}}={\vect{\hat{f}}}
  \label{eqn:NavierStokesequationALE2general}
\end{equation}
where ${\vect{\hat{u}}}$ is the vector of unknown ``DOFs'', $\matr{K}$ is the
stiffness matrix, $\fnof{\vect{\hat{g}}}{{\vect{{u}}}}$ a non-linear vector
function and ${\vect{\hat{f}}}$ the forcing vector. In \eqnref{eqn:NavierStokesweakform} the only real non-linear term is represented by $\fnof{\vect{\hat{g}}}{\vect{{u}}}=\gint{\Omega}{}{\rho(\vect{u}\cdot\grad)\vect{u}\vect{v} }{\Omega}$.
If $\fnof{\vect{\hat{g}}}{\vect{u}}$ is not $\equiv\vect{0}$ then we use Newton's method \ie
\begin{equation}
  \begin{split}
    \text{1.  } & \fnof{\matr{J}}{\vect{u}_{i}}.\delta
    \vect{u}_{i} = 
    -\fnof{\vect{\psi}}{\vect{u}_{i}} \\
    \text{2.  } & \vect{u}_{i+1}=\vect{u}_{i}+\delta
    \vect{u}_{i}
  \end{split}
\end{equation}
where $\fnof{\matr{J}}{\vect{u}}$ is the Jacobian and is given by
\begin{equation}
  \fnof{\matr{J}}{\vect{u}}=\matr{K}+
    \delby{\fnof{\vect{\hat{g}}}{\vect{u}}}{\vect{u}}
\end{equation}
with the stiffness matrix $\matr{K}$ derived from \eqnref{eqn:NavierStokesequationALE2general} by applying Green's theorem as follows:
\begin{equation}
  \begin{split}
  \matr{K}\vect{\hat{u}}=
  \gint{\Omega}{}{\grad\cdot\vect{v}p}{\Omega}
  -\gint{\Omega}{}{\mu\grad\vect{v}:\grad\vect{u}}{\Omega}
  -\gint{\Omega}{}{\rho(\vect{u}^*\cdot\grad)\vect{u}\vect{v}}{\Omega}
  +\gint{\Omega}{}{\grad\cdot \vect{u}q}{\Omega}
  \end{split}
  \label{eqn:NavierStokesweakform2}
\end{equation}
and $\fnof{\vect{\psi}}{\vect{\hat{u}}}=\matr{K}\vect{\hat{u}}+\fnof{\vect{\hat{g}}}{\vect{u}}+\vect{\hat{f}}$.


%%% Local Variables: 
%%% mode: latex
%%% TeX-master: "../../OpenCMISSNotes"
%%% End: 
