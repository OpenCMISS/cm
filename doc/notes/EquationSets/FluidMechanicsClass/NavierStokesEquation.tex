\subsection{Navier-Stokes Equations} 

\subsubsection{Governing equations:}

The Navier-Stokes equations arise from applying Newton's second law to fluid motion, i.e. the temporal and spatial fluid inertia is in equilibrium with internal (volume/body)  and external (surface) forces. The reaction of surface forces can be described in terms of the fluid stress as the sum of a diffusing viscous term, plus a pressure term. A solution of the Navier-Stokes equations is called a flow field, i.e. velocity and pressure field, which is a description of the fluid at a given point in space and time.  In the common case of an incompressible Newtonian fluid, the nonlinear Navier-Stokes equations (three-dimensional, transient) can be written using 'primitive variables' (i.e. u-velocity, p-pressure) as:
\begin{equation}
  \label{eqn:NSE}
    \rho\delby{\vect{u}}{t}+\rho(\vect{u}\cdot\grad)\vect{u}=\vect{f}-\grad{p}+\mu\laplacian{\vect{u}}
\end{equation}
accompanied by the conservation of mass (incompressibility)
\begin{equation}
  \label{eqn:NSEIncompressibility}
  \diverg{\vect{u}}=0
\end{equation}
where $\vect{u}(\vect{x},t)=(u_1,u_2,u_3)^T$ is the velocity vector depending on spatial coordinates $\vect{x}=(x_1,x_2,x_3)^T$ and the time $t$, $p$ is the scalar pressure, $\vect{f}$ an applied body force, and the material parameters $\mu$ and $\rho$ are the fluid viscosity and density, respectively.The first term represents unsteady accelerative inertial contributions, the second represents the nonlinear convective acceleration terms, the $\grad{p}$ term the pressure contributions, and the last term represents viscous stresses in the system.
% incompressibility and the LBB condition
As with Stokes flow, the incompressibility condition $\diverg{\vect{u}}=0$ also creates restrictions on the formulation known as the Ladyzhenkaya, Babuska, and Brezzi(LBB) or inf-sup consistency condition. Several methods have been devised to define a pressure function that is consistent with the velocity space using primitive variables. These include mixed element methods, penalty methods, generalized petrov-galerkin methods using pressure poisson correction, operator splitting, and semi-implicit pressure correction (Chung).

Using a classic Galerkin formulation, mixed methods are perhaps conceptually the most straightforward method of satisfying LBB, in which velocity is defined over a space one order higher than pressure (e.g. quadratic elements for velocity, linear for pressure), allowing incompressibility to be satisfied. It should be noted that our use of a mixed formulation to satisfy LBB will also be reflected in the shape functions that our weak formulation depends on. For example, using 2D elements with biquadratic velocity and linear pressure, we will have 9 weight functions for each velocity component and 4 for the pressure. 
% talk about the meaning of pressure in NSE.

\subsubsection{Weak Formulation}
% Galerkin formulation
The weak form of equations \eqref{NSE} and \eqref{NSEIncompressibility} can be found by applying standard Galerkin test functions $\vect{w}$:
\begin{equation}
 \label{BasicGalerkinNSE}
  \gint{\Omega}{}{\pbrac{\rho\delby{\vect{u}}{t}
    +\rho\vect{u}\cdot\grad{\vect{u}}
    -\vect{f}
    -\mu\laplacian{\vect{u}}
    +\grad{p}}{\vect{w}}}{\Omega} = 0
\end{equation}

Integrating by parts, we will get the weak form of the system of PDEs with the associated natural boundary conditions at the boundary $\Gamma_N$:
\begin{multline}
 \label{GalerkinNSE}
  \gint{\Omega}{}{\rho\delby{\vect{u}}{t}{\vect{w}}}{\Omega}
 +\gint{\Omega}{}{\rho\vect{u}\cdot\grad\vect{u}{\vect{w}}}{\Omega}
 +\gint{\Omega}{}{p\grad{\vect{w}}}{\Omega}
 -\gint{\Omega}{}{\mu\grad{\vect{u}}\cdot\vect{\grad{w}}}{\Omega}=\\
 \gint{\Omega}{}{\vect{f}\vect{w}}{\Omega}
 +\gint{\Gamma_N}{}{\mu\dotprod{\grad\vect{u}}{\vect{n}}\vect{w}}{\Gamma_N}
 -\gint{\Gamma_N}{}{\dotprod{p}{\vect{n}}\vect{w}}{\Gamma_N}
\end{multline}

% Boundary conditions
For more extensive discussion of this procedure, along with other weak forms of the PDEs, we refer to (Gresho and Sani). From this weak form, we see natural (Neumann) boundary conditions arising as a direct result of the integration. Neumann boundary conditions will therefore consist of a pressure term and viscous stress acting normal to a given boundary. 
\begin{gather}
 \label{eqn:NSENeumannBC}  
   \fnof{\vect{q}}{\vect{x},t} = \mu\dotprod{\grad{\fnof{\vect{u}}{\vect{x},t}}}{\normal}-p\cdot{\normal}  \\
%   \fnof{\vect{q_2}}{\vect{x},t} = p\cdot{\normal} \\
   \quad \vect{x}\in\Gamma_{N}
\end{gather}
Specification of Neumann boundaries will therefore require the specification of the viscous stress and pressures across lower order element DOFs using the mixed formulation. Dirichlet boundary conditions on a boundary $\Gamma_D$ for velocity or pressure will take the form:
\begin{gather}
 \label{eqn:NSEDirichletBC} 
  \fnof{\vect{u}}{\vect{x},t} = \fnof{\vect{d}}{\vect{x},t} \quad \vect{x}\in\Gamma_{D}\\
  \fnof{p}{\vect{x},t} = \fnof{\vect{d}}{\vect{x},t} \quad \vect{x}\in\Gamma_{D}
\end{gather}
Which may be directly applied to the relevant DOFs.

\subsubsection{Tensor notation}
Assuming no forcing terms and substituting the natural boundary as defined above, equation \eqref{GalerkinNSE} in tensor notation becomes:
\begin{multline}
 \label{TensorNSE}
  \gint{\Omega}{}{\rho\dot{u}_{i}w_{i}}{\Omega}
 +\gint{\Omega}{}{{\rho}G^{jk}u_{j}\covarderiv{u_{i}}{k}w_{i}}{\Omega}
 +\gint{\Omega}{}{G^{jk}{p_{i}}\covarderiv{w_{i}}{k}}{\Omega}\\
 -\gint{\Omega}{}{{\mu}G^{jk}\covarderiv{u_{i}}{j}\covarderiv{w_{i}}{k}}{\Omega}
 -\gint{\Gamma_N}{}{{q_{i}}{w_i}}{\Gamma_N}=\vect{0}
\end{multline}
or
\begin{multline}
 \label{Tensor2NSE}
  \gint{\Omega}{}{\rho\dot{u}_{i}w_{i}}{\Omega}
 +\gint{\Omega}{}{{\rho}G^{jk}{u_j}\pbrac{\partialderiv{u_{i}}{k}-\christoffel{i}{k}{h}u_h}{w_{i}}}{\Omega}
 +\gint{\Omega}{}{G^{jk}{p_{i}}\pbrac{\partialderiv{w_{i}}{k}-\christoffel{i}{k}{h}w_h}}{\Omega}\\
 -\gint{\Omega}{}{{\mu}G^{jk}\pbrac{\partialderiv{u_{i}}{j}-\christoffel{i}{j}{h}u_h}\pbrac{\partialderiv{w_{i}}{k}-\christoffel{i}{k}{h}w_h}}{\Omega}
 -\gint{\Gamma_N}{}{{q_{i}}{w_i}}{\Gamma_N}=\vect{0}
\end{multline}


where $G^{jk}$ is the contravariant metric tensor and
$\christoffel{i}{j}{k}$ is the Christoffel symbol of the second kind for the spatial coordinates.

\subsubsection{Finite Element Formulation}

We can now discretise the domain into finite elements \ie $\Omega=\displaystyle{\bigcup_{e=1}^{E}}\Omega_{e}$ with $\Gamma=\displaystyle{\bigcup_{f=1}^{F}}\Gamma_{f}$. \Eqnref{Tensor2NSE} now becomes:

\begin{multline}
 \label{FEMNSE}
  \dsum_{e=1}^{E}\gint{\Omega}{}{\rho\dot{u}_{i}w_{i}}{\Omega}
 +\dsum_{e=1}^{E}\gint{\Omega}{}{{\rho}G^{jk}{u_j}\pbrac{\partialderiv{u_{i}}{k}-\christoffel{i}{k}{h}u_h}{w_{i}}}{\Omega}
 +\dsum_{e=1}^{E}\gint{\Omega}{}{G^{jk}{p_{i}}\pbrac{\partialderiv{w_{i}}{k}-\christoffel{i}{k}{h}w_h}}{\Omega}\\
 -\dsum_{e=1}^{E}\gint{\Omega}{}{{\mu}G^{jk}\pbrac{\partialderiv{u_{i}}{j}-\christoffel{i}{j}{h}u_h}\pbrac{\partialderiv{w_{i}}{k}-\christoffel{i}{k}{h}w_h}}{\Omega}
 -\dsum_{f=1}^{F}\gint{\Gamma_N}{}{{q_{i}}{w_i}}{\Gamma_N}=\vect{0}
\end{multline} 
or


\begin{multline}
 \label{FEMNSE2}
  \dsum_{e=1}^{E}\gint{\Omega}{}{\rho\dot{u}_{i}w_{i}}{\Omega}
 +\dsum_{e=1}^{E}\gint{\Omega}{}{{\rho}G^{jk}{u_j}\pbrac{\partialderiv{u_{i}}{k}-\christoffel{i}{k}{h}u_h}{w_{i}}}{\Omega}
 +\dsum_{e=1}^{E}\gint{\Omega}{}{G^{jk}{p_{i}}\pbrac{\partialderiv{w_{i}}{k}-\christoffel{i}{k}{h}w_h}}{\Omega}\\
 -\dsum_{e=1}^{E}\gint{\Omega}{}{{\mu}G^{jk}\pbrac{\partialderiv{u_{i}}{j}-\christoffel{i}{j}{h}u_h}\pbrac{\partialderiv{w_{i}}{k}-\christoffel{i}{k}{h}w_h}}{\Omega}=\\
 \dsum_{f=1}^{F}\gint{\Gamma_N}{}{{\mu}G^{jk}\pbrac{\partialderiv{u_{i}}{k}-\christoffel{i}{k}{h}u_h}{n_{i}}{w_{i}}}{\Gamma_N}
-\dsum_{f=1}^{F}\gint{\Gamma_N}{}{G^{jk}{p}{n_i}{w_i}}{\Gamma_N}
\end{multline} 


We will assume that the dependent variables $\vect{u}$ and $p$ can be
interpolated separately in space and time. Here we must also be
careful to note again the discrepancy between the functional
spaces for velocity and pressure using a mixed formulation to satisfy the LBB
consistency requirement. We will therefore define two separate weighting functions: for the velocity space on $\Omega$, the test function will be $\psi_i$ and for the pressure space, $\phi_i$,
giving:

\begin{gather}
  \fnof{\vect{u}}{\vect{x},t}=\gbfn{n}{}{\vect{x}}\fnof{\nodept{\vect{u}}{n}}{t}\\
  \fnof{p}{\vect{x},t}=\phi_{n}({\vect{x}})\fnof{\nodept{p}{n}}{t}
\end{gather}

In standard interpolation notation within an element, this will
become:

\begin{gather}
  \fnof{u_{i}}{\vect{\xi},t}=\idxgbfn{i}{n}{\beta}{\vect{\xi}}
  \fnof{\idxnodedof{u}{i}{n}{\beta}}{t}\idxgsf{i}{n}{\beta}\\
  \fnof{p_{}}{\vect{\xi},t}=\phi_{in}^{\beta}({\vect{\xi}})
  \fnof{\idxnodedof{p}{}{n}{\beta}}{t}\idxgsf{i}{n}{\beta}
\end{gather}

where $\fnof{\idxnodedof{u}{i}{n}{\beta}}{t}$ are the time varying nodal
degrees-of-freedom for velocity component $i$, node $n$, global
derivative $\beta$
$\idxgbfn{i}{n}{\beta}{\vect{\xi}}$ are the corresponding velocity basis functions 
and $\idxgsf{i}{n}{\beta}$ are the scale factors. The scalar pressure
DOFs, $\fnof{\idxnodedof{p}{}{n}{\beta}}{t}$ are interpolated
similarly.

For the neumann boundary, we will again separate $q_i$ into its
component velocity and pressure terms as noted in
\eqnref{eqn:NSENeumannBC}. Interpolating based on the velocity and
pressure basis functions above, 

\begin{equation}
  \fnof{q_{i}}{\vect{\xi},t} &= \idxgbfn{i}{o}{\gamma}{\vect{\xi}}
\end{equation}

Here $N_{u}$ represents the total $u$ velocity nodes for a given dimension and $N_p$ the
total number of $p$ pressure nodes. Our finite element formulation for
the mixed formulation becomes:

\begin{multline}
 \label{FEMNSE}
  \dsum_{i=1}^{N_u}\gint{\Omega}{}{\rho\dot{u}_{i}w_{i}}{\Omega}
 +\dsum_{i=1}^{E}\gint{\Omega}{}{{\rho}G^{jk}{u_j}\pbrac{\partialderiv{u_{i}}{k}-\christoffel{i}{k}{h}u_h}{w_{i}}}{\Omega}
 +\dsum_{i=1}^{E}\gint{\Omega}{}{G^{jk}{p_{i}}\pbrac{\partialderiv{w_{i}}{k}-\christoffel{i}{k}{h}w_h}}{\Omega}\\
 -\dsum_{i=1}^{E}\gint{\Omega}{}{{\mu}G^{jk}\pbrac{\partialderiv{u_{i}}{j}-\christoffel{i}{j}{h}u_h}\pbrac{\partialderiv{w_{i}}{k}-\christoffel{i}{k}{h}w_h}}{\Omega}
 -\dsum_{f=1}^{F}\gint{\Gamma_N}{}{{q_{i}}{w_i}}{\Gamma_N}=\vect{0}
\end{multline}



% convective Petrov-Galerkin formulation
For convective flows, we augment the convective term of \eqref{GalerkinNSE} with a Petrov-Galerkin test function of the form $\vect{\Phi}=\vect{w}+\vect{\Psi}$, where $\vect{w}$ represent standard Galerkin weights as used in \eqref{GalerkinNSE} and $\vect{\Psi}$ takes the form:
\begin{equation}
 \label{PetrovGalerkinTest}
  \vect{\Psi}=\frac{\alpha{L}}{2}
\end{equation}


%ALE form
Whereas \eqnref{eqn:NavierStokesequation1} has been formulated in Eulerian form, moving domain approaches often require the ALE modification taking an additional term into account, depending on the fluid density $\rho$ and a correction velocity $\vect{u}^*$ which yields to:
\begin{equation}
    \rho((\vect{u}-\vect{u}^*)\cdot\grad)\vect{u}=\vect{f}-\grad{p}+\mu\laplacian{\vect{u}}
  \label{eqn:NavierStokesequationALE}
\end{equation}
So far, the nonlinear term in \eqnref{eqn:NavierStokesequation1} represents the fluid spatial acceleration only. \eqnref{eqn:NavierStokesequation2} now also takes the dynamic inertia terms into account
\begin{equation}
    \rho\delby{\vect{u}}{t}+ \rho((\vect{u}-\vect{u}^*)\cdot\grad)\vect{u}=\vect{f}-\grad{p}+\mu\laplacian{\vect{u}}
  \label{eqn:NavierStokesequation2}
\end{equation}
which gives us the complete Navier-Stokes equations in ALE formulation.
The following section, however, describes the reordered quasi-static formulation of  \eqnref{eqn:NavierStokesequationALE}:
\begin{equation}
\rho((\vect{u}-\vect{u}^*)\cdot\grad)\vect{u}-\mu\laplacian{\vect{u}}+\grad{p}=\vect{f}
%     -\grad{p}+\mu\laplacian{\vect{u}}-\rho(\vect{u}^*\cdot\grad)\vect{u}=\vect{f}
  \label{eqn:NavierStokesequationALE2}
\end{equation}

%\subsubsection{Weak formulation:}

The corresponding weak form of the equation system consisting of \eqnref{eqn:NavierStokesequation1} and \eqnref{eqn:NavierStokesmasequation} can be written in the general dynamic form (see section 2.3.2)
\begin{equation}
  \matr{M}\fnof{\ddot{\vect{u}}}{t}+\matr{C}\fnof{\dot{\vect{u}}}{t}+\matr{K}\fnof{\vect{u}}{t}+
  \fnof{\vect{g}}{\fnof{\vect{u}}{t}}+\fnof{\vect{f}}{t}=\vect{0}
  \label{eqn:generaldynamicnonlinear}
\end{equation}
where u(t) is the dependent variables vector $\vect{u}(\vect{x},t)$ and $p$ for the degrees of freedom. $\matr{M}$ is the mass matrix, which provides the shape function based weights, $\matr{C}$ is the transient damping matrix (which we will discuss further below). $\matr{K}$ represents the stiffness matrix, which will contain the linear parts of the operator, including the viscous terms, the conservation of mass terms, and pressure terms. $\fnof{\vect{g}}{\fnof{\vect{u}}{t}}$ is the nonlinear vector function for the convective terms and $\fnof{\vect{f}}{t}$ the forcing vector. 

The corresponding weak form of the equation system consisting of \eqnref{eqn:NavierStokesequation1} and \eqnref{eqn:NavierStokesmasequation} can be written as:
\begin{equation}
  \gint{\Omega}{}{\rho(\vect{u}\cdot\grad)\vect{u}\vect{v} }{\Omega}
  -\gint{\Omega}{}{\rho(\vect{u}^*\cdot\grad)\vect{u}\vect{v} }{\Omega}
  +\gint{\Omega}{}{\mu\laplacian{\vect{u}}\vect{v}}{\Omega}
  -\gint{\Omega}{}{\grad{p}\vect{v}}{\Omega}
  +\gint{\Omega}{}{\grad\cdot \vect{u}q}{\Omega}=
  \gint{\Omega}{}{\vect{f}\vect{v}}{\Omega}  
  \label{eqn:NavierStokesweakform}
\end{equation}
The general form for this kind of equation system is
\begin{equation}
  \matr{K}{\vect{\hat{u}}}+
  \fnof{\vect{\hat{g}}}{{\vect{{u}}}}={\vect{\hat{f}}}
  \label{eqn:NavierStokesequationALE2general}
\end{equation}
where ${\vect{\hat{u}}}$ is the vector of unknown ``DOFs'', $\matr{K}$ is the
stiffness matrix, $\fnof{\vect{\hat{g}}}{{\vect{{u}}}}$ a non-linear vector
function and ${\vect{\hat{f}}}$ the forcing vector. In \eqnref{eqn:NavierStokesweakform} the only real non-linear term is represented by $\fnof{\vect{\hat{g}}}{\vect{{u}}}=\gint{\Omega}{}{\rho(\vect{u}\cdot\grad)\vect{u}\vect{v} }{\Omega}$.
If $\fnof{\vect{\hat{g}}}{\vect{u}}$ is not $\equiv\vect{0}$ then we use Newton's method \ie
\begin{equation}
  \begin{split}
    \text{1.  } & \fnof{\matr{J}}{\vect{u}_{i}}.\delta
    \vect{u}_{i} = 
    -\fnof{\vect{\psi}}{\vect{u}_{i}} \\
    \text{2.  } & \vect{u}_{i+1}=\vect{u}_{i}+\delta
    \vect{u}_{i}
  \end{split}
\end{equation}
where $\fnof{\matr{J}}{\vect{u}}$ is the Jacobian and is given by
\begin{equation}
  \fnof{\matr{J}}{\vect{u}}=\matr{K}+
    \delby{\fnof{\vect{\hat{g}}}{\vect{u}}}{\vect{u}}
\end{equation}
with the stiffness matrix $\matr{K}$ derived from \eqnref{eqn:NavierStokesequationALE2general} by applying Green's theorem as follows:
\begin{equation}
  \begin{split}
  \matr{K}\vect{\hat{u}}=
  \gint{\Omega}{}{\grad\cdot\vect{v}p}{\Omega}
  -\gint{\Omega}{}{\mu\grad\vect{v}:\grad\vect{u}}{\Omega}
  -\gint{\Omega}{}{\rho(\vect{u}^*\cdot\grad)\vect{u}\vect{v}}{\Omega}
  +\gint{\Omega}{}{\grad\cdot \vect{u}q}{\Omega}
  \end{split}
  \label{eqn:NavierStokesweakform2}
\end{equation}
and $\fnof{\vect{\psi}}{\vect{\hat{u}}}=\matr{K}\vect{\hat{u}}+\fnof{\vect{\hat{g}}}{\vect{u}}+\vect{\hat{f}}$.


%%% Local Variables: 
%%% mode: latex
%%% TeX-master: "../../OpenCMISSNotes"
%%% End: 
